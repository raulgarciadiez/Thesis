\pagestyle{empty}
\noindent
\section*{Acknowledgments}

I would like to use this opportunity to thank all the people that have assisted me during these years to reach my goals and have contributed to the conclusion of this thesis. Though I tried to avoid any verbose language during the main text, it will be extremely challenging to remain synthetic when acknowledging the contributions of the following people.
\vspace{2ex}

\noindent First of all, I would like to thank \emph{Dr. Michael Krumrey}, the leader of the working group Röntgenradiometrie of the Physikalisch-Technische Bundesanstalt (PTB) and the person who provided me the proper human and scientific environment to perform successful experiments and pursue my research interests. Under his leadership, I could concentrate in the relevant aspects of my investigations and focus all my energy into my research.
\vspace{2ex}

\noindent I am very grateful also to \emph{Prof. Dr. Mathias Richter} for giving me the opportunity to participate on the activity of the PTB in BESSY II and encourage me to chase my scientific goals. His motivation and constructive advices during these years have been really helpful and are highly appreciated. 
\vspace{2ex}

\noindent \emph{Prof. Dr. Stefan Eisebitt} and \emph{Prof. Dr. Simone Raoux} are also kindly acknowledged for the precious advice given to complete my research work and for the concern to read and prove this written thesis. Their many research interests inspired me to find new alternatives to old scientific problems.
\vspace{2ex}

\noindent I am greatly indebted to my mentor \emph{Dr. Christian Gollwitzer} for his supervision and honest interest throughout these last 4 years. The valuable scientific expertise he provided me with cannot overshadow the great moments we spent together in the laboratory. Without his support and expert advise, the completion of this thesis would have been virtually impossible.
\vspace{2ex}

\noindent And also my most sincerely acknowledgement to the whole Arbeitsgruppe 7.11 of PTB, whose individuals have contributed to my work both technically and personally. I am especially thankful to all the engineers who have provided the technical support to perform SAXS experiments in an outstanding way. During these years, the group line-up included \emph{Levent Cibik, Ulf Knoll, Stefanie Langner, Swenja Schreiber, Layla Riemann} and \emph{Peter Müller}.
\vspace{2ex}

\noindent I don't want to forget the many graduate students and postdocs with whom I have crossed paths in PTB and who have influenced and enhanced my research like \emph{Dr. Jan Wernecke, Anal\'{i}a Fern\'{a}ndez Herrero, Anton Haase, Mika Pflüger, Oleksey Mariasov} and \emph{Dr. Victor Soltwisch}.
\vspace{2ex}

\noindent I am glad to acknowledge also the excellent job that all the members of the Laboratory of the PTB in BESSY II perform day after day as well as the Helmholtz-Zentrum Berlin (HZB) scientists who operate the synchrotron facility. Without the continuous and stable performance of BESSY II, most of the experimental data shown in this thesis could not have been collected.
\vspace{2ex}

\noindent I am in debt both personally and scientifically with \emph{Dr. Zoltan Varga} from the Institute of Materials and Environmental Chemistry (Research Centre for Natural Sciences, Budapest, Hungary). His expertise in SAXS and liposomal structures is unparalleled and some of the ideas presented in this thesis derive directly from fruitful discussions with him. Besides he prepared the empty liposomes described in chapter \ref{chap:bio_applications} and encouraged me to give Caelyx\textregistered\ a chance.

\vspace{2ex}

\noindent I want to acknowledge the long-term and rewarding collaboration established with the Surface and Nanoanalysis group of the National Physical Laboratory (Teddington, UK) led by \emph{Dr. Alex Shard}. I want to thank especially \emph{Dr. Caterina Minelli} for sharing her extensive knowledge about polymeric colloids and bio-surfaces and for the preparation of the protein-coated nanoparticles employed in chapter \ref{chap:bio_applications} as well as I want to highlight our common interest in characterization techniques for low-density nanoparticles as observed throughout this thesis. I am also very grateful to \emph{Dr. Aneta Sikora} for her predisposition and competence on the DCS measurements presented in chapter \ref{chap:simultaneous_size_density}.
\vspace{2ex}

\noindent I would like to thank sincerely \emph{Dr. Armin Hoell} from HZB for the continuous collaboration with the HZB SAXS setup, which was used in the majority of experiments presented here. His expertise in (A)SAXS has been inspiring and his knowledge of the SAXS apparatus has proven very beneficial for my research.
\vspace{2ex}

\noindent I am grateful to \emph{Eike Gericke} from Humboldt Universität Berlin for his interest in new calibration standard materials for SAXS, who brought me in contact with other alternatives to AgBehe. I would like to thank also \emph{Roland Schmack} from Technische Universität Berlin for the synthesis of the SBA-15 sample used in the study of chapter \ref{chap:experimental_setup}.
\vspace{2ex}

\noindent It is important for me to mention the personal and scientific support given by other HZB colleagues like \emph{Dr. Kaan Atak} and \emph{Dr. Wilson Quevedo}, whose scientific comments shaped this thesis in subtle but vital ways. The proofreading job of Dr. Atak is also very valuable and it is only comparable to the proofreading effort provided by \emph{Dr. Marc Cano-Bret} from the Department of Physics and Astronomy of the Shanghai Jiao Tong University (China), whose long-lasting friendship and motivation pushed me through some difficult moments.
\vspace{2ex}

\noindent A very special "thank you" to all my physics and non-physics friends in Berlin, \emph{Markus, Alexander, Alessandro, Alisio, Alberto, Fede, Mattia, Donal} and the \emph{Margon's 12} crew in Barcelona, who make a good day out of a bad one.
\vspace{2ex}

\noindent And of course, a warm and honest thanks to my family in general and to my parents and Marta in particular. They have been all the strength that I needed in the darkest moments in Berlin. \emph{Muchas gracias por saber ser el apoyo que he necesitado en los momentos m\'{a}s duros}.

\cleardoublepage
