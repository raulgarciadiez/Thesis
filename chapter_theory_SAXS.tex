\chapter{Theoretical Background}
\label{chap:theory_SAXS}
In this chapter, the basic phyisical principles underlying the operation of small-angle X-ray scattering are presented, focusing principally on the interaction between X-rays and matter and the elastic scattering of X-rays by an ensemble of electrons. An entire section is devoted to the theoretical framework used in contrast variation experiments in SAXS, where concepts suchs as the isoscattering point and the Guinier radius are introduced.

\section{Interaction of X-rays and matter}

X-rays are electromagnetic radiation which propagates in vacuum along the direction of the wavevector $\vec{k}$. The incident X-ray radiation can be described by the wave function of a monochromatic plane wave:

\begin{equation}
        \label{eq:IncidentWave}
        \Psi_0\left( \vec{r} \right)=A_0 e^{i \vec{k}\vec{r} }
\end{equation}

where the wavenumber $k=\abs{\vec{k}}$ is related to the X-ray wavelength by $k=\sfrac{2\pi}{\lambda}$. Conventially, X-ray wavelengths ranges between 0.01 and 10 nm, although SAXS experiments are conducted normally at the hard X-ray range, e.g. at wavelenghts between 0.02 and 0.8 nm. Due to the wave-particle duality of electromagnetic radiation, X-rays possess a particle nature as well, represented by the quantisation of light into an ensemble of photons with an energy $\hbar \omega$. The photon energy is related to the X-ray wavelength by\textcolor{red}{REF}

\begin{equation}
        \lambda = \frac{h c}{E_{ph}}
\end{equation}

where $h=2\pi\hbar$ is the Planck's constant and $c$ is the speed of light in vacuum. The photon energies employed typically in SAXS experiments stretch between the silicon K-edge at 1.7 keV and some dozens of keV, including the classical copper K$_{\alpha}$ emission line at 8 keV.

\subsection{Beer-Lambert law}
\label{sec:BeerLambert}

\begin{figure}%[htbp]
	\centering
	        \def\svgwidth{0.75\linewidth}
		\input{Figures/BeerLambertScheme.pdf_tex}
		\caption{The Beer-Lamber law: The attenuation of X-rays through a medium of thickness $d$ is depicted.}
		\label{fig:BeerLambertScheme}
\end{figure}

The interaction of X-ray photons and matter produce an attenuation of the incident radiation intensity $I_0$ which is related to the properties and volume of the material. The decrease of the intensity through a medium is schematically depicted in figure \ref{fig:BeerLambertScheme} and described by the Beer-Lambert law\textcolor{red}{REF}:

\begin{equation}
        I\left( x \right)=I_0e^{-\mu x}
\end{equation}

where $\mu$ is the linear attenuation coefficient and $x$ is the radiation path length. The attenuation coefficient is dependent on the material composition and the photon energy and is directly related to the extinction coefficient $\beta$, e.g. the imaginary part of the refraction index $n$, by\textcolor{red}{REF} 

\begin{equation}
        \mu (E) = \frac{4\pi}{hc} E \beta(E)
\end{equation}

Considering that the refractive index is expressed generally by $n = 1 - \delta +i \beta$\textcolor{red}{REF} and $\delta<10^{-3}$ in the X-ray regime \textcolor{red}{REF}, refraction effects can be neglected in scattering experiments because $\Re(n)$ is very close but smaller than unity.

When the attenuating medium is composed of different atomic species, $\mu$ can be expressed as the summation of each component attenuation coefficient:

\begin{equation}
        \mu = \sum_i \mu^i = \sum_i \rho_e^i \sigma^i  =N_A\sum_i  \frac{Z^i}{A^i} \rho^i \sigma^i
\end{equation}

where $N_A$ is the Avogadro constant, $\sigma$ is the attenuation cross-section and $\rho_e$ is the number density of absorbing centers. The cross-section $\sigma$ is defined as the effective area in which photon-matter events occur. In the X-ray regime, photons interact principally with the atomic electrons, thus $\rho_e$ is the electron density and is directly proportional to the atomic number $Z$, the atomic mass number $A$ and the physical density $\rho$ of the component $i$. 

\begin{figure}%[htbp]
	\centering
		\input{Figures/AttenuationWater}
		\caption{The different contributions to the attenuation of water at room temperature and the total attenuation.}
		\label{fig:AttenuationWater}
\end{figure}

In fact, the attenuation cross-section $\sigma$ is dependent upon the several different mechanisms in which a X-ray photon interacts with the atomic electrons. The 3 most relevant effects are the photoelectron absorption, the coherent scattering and the incoherent scattering, which sum up to the total attenuation coefficient:

\begin{equation}
        \mu = \rho_e (\tau_{\text{abs}}+\sigma_{\text{scat, coh}}+\sigma_{\text{scat, incoh}})
\end{equation}

When the X-ray photon is completely absorbed by the atom, the event is called photoelectron absorption because a photoelectron with the excess energy is expelled from an inner atomic shell, leaving the atom ionized. The created core-hole is consequently filled by an electron from an outer shell either by a radiative process, i.e. \emph{fluorescence}, or by a non-radiative mechanism emitting a secondary electron, i.e. Auger effect. The photoelectric effect is the predominant contribution to the attenuation principally at low X-ray energies and the ultraviolet regime, as shown in figure \ref{fig:AttenuationWater}. 

The other relevant contributions at the X-ray range are related to scattering processes. In an inelastic scattering event, the energy of the incident photon is partially transfered to a loosely bond electron resulting in a scattered photon with a lower frequency, according to the Compton relation $\Delta\lambda = \sfrac{h}{m_e c}\left( 1 - \cos{\Theta} \right) $, where $\Theta$ is the scattering angle. The Compton scattering is incoherent and contributes generally less than the elastic scattering at energies below 10 keV, as observed in figure \ref{fig:AttenuationWater}. Besides, the coherent scattering signal is the summation of the constructive interferences of the electromagnetic wave, which produces a higher scattered intensity than the inelastic scattering. In fact, the elastic scattering of X-rays is the main proccess used in material investigations and the pyhsical principle behind SAXS.

\subsection{Elastic scattering}

When the wavelength of the scattered wave is the same than that of the incident one, the proccess is named elastic scattering or coherent scattering and the resulting intensity is the absolute square of the sum of the scattering amplitudes. In the following sections, the elastic scattering theory will be presented for the classical case and for an ensemble of electrons.

\subsubsection{Thomson scattering}

Classically, the elastic scattering of a photon by a free electron is described by the conservation of the photon energy, i.e. the wavenumber of the scattered wave is the same than the incident one ($\abs{\vec{k_{\text{s}}}}=\abs{\vec{k}}$). Consequently for unpolarized incident radiation, the intensity of the scattered wave at a distance $r$ and with a scattering angle $2\Theta$ is defined by:

\begin{equation}
        I_{\text{scat}}\left( r,\Theta \right)= I_0 \left( \frac{r_e}{r} \right) ^2 \left( \frac{1+\cos^2{2\Theta}}{2} \right)
\end{equation}

where $r_e= \sfrac{e^2}{4\pi\epsilon_0 m c^2}=2.82\cdot10^{-15}$ m is the Thomson or classical electron radius. A relevant quantity in scattering processes is the differential scattering cross-section $\sfrac{d\sigma}{d\Omega}$, which is defined as the the number of scattered photons per time and per solid angle over the incident intensity per time and per area:

\begin{equation}
        \label{eq:thomson_cross_section}
        \frac{d\sigma}{d\Omega}= \frac{I_{\text{scat}} \cdot \left(r^2 \Delta \Omega \right)}{I_0\Delta \Omega}=r_e^2\left( \frac{1+\cos^2{2\Theta}}{2} \right)
\end{equation}

where $r^2 \Delta \Omega$ is the detector surface in the plane of the impact parameter. The total Thomson scattering cross-section is $\sigma = \sfrac{8\pi r_0^2}{3} = 0.665\cdot10^{-24}$ cm$^2$ and similarly to $\sfrac{d\sigma}{d\Omega}$ is both proportional to $r_e^2$ and independent on the photon energy at the X-ray regime away of a resonant excitation.

\subsubsection{Scattering by an ensemble of electrons}

\begin{figure}%[htbp]
	\centering
	        \def\svgwidth{0.75\linewidth}
		\input{Figures/BeerLambertScheme.pdf_tex}
		\caption{Scheme of an scattering event at a position $\vec{r}$}
		\label{fig:FraunhoferScheme}
\end{figure}

The scattering of a photon by an ensemble of weak bounded electrons can be studied by considering the interaction of particles with a three-dimensional weak potential field $\phi(\vec{r})$. The resulting wave can be expressed as a linear combination of the incidient plane wave (see equation \ref{eq:IncidentWave}) and the scattered spherical wave at the position $\vec{r}$:

\begin{equation}
       \Psi\left( \vec{r} \right)= \Psi_0\left( \vec{r} \right) +  \Psi_{\text{scat}}\left( \vec{r} \right)
\end{equation}

Introducing this expression at the time-independent Schrodinger equation and considering the scattering wave as a perturbation produced by the scattering density function $\phi(\vec{r})$ \textcolor{red}{REF}, it can derived that

\begin{equation}
       \Psi_{\text{scat}}\left( \vec{r} \right)=C \int \frac{e^{ i \vec{k}\abs{\vec{r}-\vec{r}'} } } {\abs{\vec{r}-\vec{r}'}} \phi(\vec{r}')   \Psi\left( \vec{r}' \right) d\vec{r}'^3
\end{equation}

where $C$ is the so-called \emph{scattering length}. If the detection position $\vec{r}$ is at distance much larger than the scattering object size, as outlined in figure \ref{fig:FraunhoferScheme}, the Fraunhoffer approximation\textcolor{red}{REF} applies and $\abs{\vec{r}-\vec{r'}}=r$, resulting in

\begin{equation}
       \Psi_{\text{scat}}\left( \vec{r} \right)=C \frac{e^{i k r}}{r} \int e^{ -i \vec{k}\vec{r}' }  \phi(\vec{r}')   \Psi\left( \vec{r}' \right) d\vec{r}'^3
\end{equation}

Assuming that there are no multiple scattering events due to the low concentration of scatterers and that the potential field is weak, the first Born approximation\textcolor{red}{REF} can be employed ($ \Psi\left( \vec{r} \right) \simeq \Psi_0\left( \vec{r} \right)$), leading to

\begin{equation}
       \Psi_{\text{scat}}\left( \vec{r} \right)=C A_0 \frac{e^{i k r}}{r} \int e^{ i \vec{q}\vec{r}' }  \phi(\vec{r}')  d\vec{r}'^3
\end{equation}

where $\vec{q}=\vec{k_s} - \vec{k}$ is the momentum transfer vector. Analogously to equation \ref{eq:thomson_cross_section}, the differential scattering cross-section is:

\begin{equation}
        \label{eq:rayleigh_cross_section}
        \frac{d\sigma}{d\Omega}= \frac{\abs{\Psi_{\text{scat}}}^2 \cdot \left(r^2 \Delta \Omega \right)}{\abs{\Psi_{0}}^2\Delta \Omega}=r_e^2 \abs{f(\vec{q})}^2
\end{equation}
where $f(\vec{q})=\int e^{ i \vec{q}\vec{r}' }  \phi(\vec{r}')  d\vec{r}'^3$ is the scattering amplitude and the scattering length is the classical electron radius $r_e$. The scattering amplitude is simply the Fourier transform of the scattering potential field $\phi(\vec{r})$. 

This type of scattering mechanism is called Rayleigh-Gans-Debye when the refractive index of the object ($n_{\text{obj}}$) is close to unity and the condition $\sfrac{2\pi}{\lambda} \cdot a \cdot n_{\text{med}} \cdot \abs{1-\sfrac{n_{\text{obj}}}{n_{\text{med}}}}\ll1$ is fulfilled, being $a$ the size of the object. For X-ray photons with wavelenghts $\lambda$ in the Angstrom range and colloidal objects in the nanoscale, this approximation can be employed. In the case of optical radiation on colloids, the Mie scattering framework is used, while the Rayleigh scattering corresponts to light wavelengths much larger than the scattering object.

\subsubsection{Anomalous scattering}

In the of an atom, the scattering amplitued of an atom is $f=Z$

This is valid when the incident radiation energy is much greater than the energy corresponding to the an energy excitation of the K, L shell of the atom

The scattering amplitued depends on the wavelength of the X-ray
\begin{equation}
        f(E) = f_0 + f'(E) + i f'' (E)
\end{equation}

$f''$ is related with attenuation coefficient by

\begin{equation}
        \mu_m=\mu \rho = \frac{2 N_A}{A} \lambda r_0 f''
\end{equation}

and f' with kramers-kronig relation ship:

\begin{equation}
        f'(\omega) = \frac{2}{\pi} \int_0^{\infty} \frac{\omega'f''(\omega')d\omega'}{\omega^2-\omega'^2}
\end{equation}

where $\lambda$ and $\omega$ are related to  photon energy


\section{Small-angle X-ray scattering}

In the case of SAXS, the scatterers are electrons and therefore the field can be considered to be the electron distribution within the object
\begin{equation}
       \phi(\vec{r})=\rho_e(\vec{r})
\end{equation}
The electron density is related to the mass density
\begin{equation}
        \rho_e = \rho N_A \frac{Z}{A}
\end{equation}
fulfilling the condition of the number of electrons in the material:
\begin{equation}
        \int \rho_e(\vec{r}')  d\vec{r}'^3 = Z
\end{equation}

\subsection{Spherical symmetry and the momentum transfer $q$}

If we consider that the object has spherical symmetry $\rho_e(\vec{r})=\rho_e(r)$ then

\begin{equation}
       f(q,r)=\frac{4\pi}{q} \int_0^{\infty} \rho_e(r') r \sin(qr')  dr'
\end{equation}

The modulus of the momentum transfer vector $q$ is defined by
\begin{equation}
       q=\abs{q}=\abs{\vec{k} - \vec{k}_0}
\end{equation}

considering elastic scattering $\abs{\vec{k}}=\abs{\vec{k}_0}=\frac{2\pi}{\lambda}$

\begin{equation}
q=\frac{2\pi }{\lambda}\sin\theta=\frac{4\pi E}{h c}\sin\theta ,
\end{equation}

where \(\theta\) is half of the scattering angle, \(h\) is the Planck constant and \(c\) is the speed of light. \textcolor{red}{GRAPH WITH SCATTERING ANGLE AND CONTINUUM OBJECT WHERE THE POSITIONS ARE DEFINED: R AND R'}
\subsubsection{Sphere form factor}
For the simple case of a solid sphere with uniform density $\rho_e(r>R)=0$ and $\rho_e(r<R)=\rho$, the integral is limited to the radius of the particle $R$

\begin{equation}
       f_{sph}(q,R)=\frac{4}{3}\pi R^3 \rho \left( 3\frac{\sin(qR)-qR\cos(qR)}{\left( qR \right)^3} \right)
\end{equation}

\subsection{Modelling of the scattering curve}
What about size distributions? Log-normal, gaussian, Monte-carlo free number of sizes (Pauw)

The scattering intensity of an ensemble of randomly oriented nanoparticles in suspension can be expressed as a function of the momentum transfer \( q \), modulus of the scattering vector \(\vec q\), as
\begin{equation}
\label{eq:intensity}
I(q)=N\int_{0}^{\infty} g(R)\left|F(q,R) \right|^2 dR,
\end{equation}
where \(N\) is the number of scatterers, \(g(R)\) is the size distribution function and \(F(R)\) is the particle form factor, which depends on the inner radial structure of the particle. If the particle shows a heterogeneous morphology, the form factor differs qualitatively for different suspending medium densities.  For sufficiently monodisperse particle suspensions, the Fourier region of the scattering curve shows pronounced minima that characterize the particle structure. 

For a typical morphology with sharp interfaces between the radial symmetric components of the particle with radius \(R_i\) the form factor is
\begin{equation}
\label{eq:multicore-shell}
F\left(q,R \right)= \Delta \eta f_{sph}(q,R)+\sum_{i=1}^{n-1} \Delta\rho_i \left( f_{sph}(q,R_{i+1})-f_{sph}(q,R_{i}) \right) ,
\end{equation}
where \(R\) is the external radius of the particle, \( n \) is the number of concentric shells and \(f_{sph}\) is the form factor of a homogeneous solid sphere given by
\begin{equation}
f_{sph}(q,R)=\frac{4}{3} \pi R^3  \left( 3\frac{\sin{qR}-qR\sin{qR}}{(qR)^3}\right).
\label{eq:ff_sph}
\end{equation}
\begin{itemize}
	\item [Sphere] Gudrun polymeric colloids???
	
	In the case of the PMMA-COOH colloids, the form factor is calculated for a homogeneous solid sphere with electron density $\rho_0$:
\begin{equation}
f_{sph}(q,R)=\frac{4}{3} \pi R^3  \left( 3\frac{\sin{qR}-qR\cos{qR}}{(qR)^3}\right)=\frac{F(q,R)}{\rho_0}.
\label{eq:ff_sph}
\end{equation}
	\item [Core-shell] Interface effects???
	
	The model represents a radially symmetric particle with a sharp interface between the outer shell and the inner core. The form factor is described by

\begin{equation}
	\begin{split}
	F(q,R)= & \Delta\eta f_{sph}(q,R)+ \\
	& \Delta\rho\left[ f_{sph}(q,R)-f_{sph}(q,R_{core}) \right] ,
	\end{split}
\label{eq:ff_cs}
\end{equation}

where \(R \) and \(R_{core} \)  are the outer shell and inner core radii respectively, the excess of electron density is \(\Delta\rho=\rho_{shell}-\rho_{core}\) and the contrast is expressed as \(\Delta\eta=\rho_{core}-\rho_{solv}\), where $\rho_{solv}$ is the electron density of the suspending medium.

	\item [Onion model] It can be used for single-SAXS experiment maybe
	\item [Vesicle] 5 gaussian????
	\item [Inclusion of background] a+b*q-4
	\item [Double shell model] Used in the protein-coated Kisker NPs (last chapter)
\end{itemize}

\subsubsection{Guinier approximation}
deviation when using too few point
Polydispersity effects

\section{Contrast variation}
Solvent variation

ASAXS

When analyzing contrast variation data, a widespread theoretical approach is based in the non-interacting model proposed by Stuhrmann $\&$ Kirste \citeyear{stuhrmann_elimination_1965,stuhrmann_elimination_1967} for monodisperse particles. The so-called \emph{basic functions} formulation differentiates, independently of the particle inner structure, the contributions which depend on the varying solvent density or contrast (\(\Delta\eta=\rho_{core}-\rho_{solv}\)) and on the excess of electron density of each component \(\Delta \rho_i =\rho_i-\rho_{core}\). 
\subsection{Isoscattering point}
One of the best known features appearing in a contrast variation experiment is the existence of \emph{isoscattering points}. At these specific \( q\)-values, the scattering intensity is independent of the adjusted solvent contrast, i.e. all scattering curves intersect in the isoscattering points regardless of the contrast. The isoscattering points \(q^{\star}\) are particularly interesting because they emerge for any spherical particle with an inner structure and a sufficiently narrow size distribution. From the contrast-depending part of equation \eqref{eq:multicore-shell}, a model-free expression can be derived which relates the position of the isoscattering points \(q^{\star}_i\) with the external radius of the particle \( R \), independent of its radial structure \cite{kawaguchi_isoscattering_1992}:
\begin{equation}
\label{eq:isoscattering}
\tan(q^{\star}_iR)=q^{\star}_iR
\end{equation}
The positions of the isoscattering points correspond to the minima positions of the scattering intensity of a compact spherical particle with radius \( R \). Although this expression is derived for the monodisperse case, it can still be applied up to a moderate degree of polydispersity, if care is taken regarding the shift of the minima position due to polydispersity \cite{beurten_polydispersity_1981}. If defining the polydispersity degree \( p_d\) as the full width half maximum of the particle size distribution divided by its average value, for size distributions with \( p_d\) larger than \( \approx 30\,\% \), the isoscattering point is not well defined and the intersection point of the curves is smeared out, showing a diffuseness in the isoscattering point position \cite{kawaguchi_isoscattering_1992}.
\subsubsection{Possible deviations}
Polydispersity and ellipticity smearing (simulation, calculation)
\subsection{Basic functions approach}
When analyzing contrast variation data, a widespread theoretical approach is based in the non-interacting model proposed by Stuhrmann $\&$ Kirste (\citeyear{stuhrmann_elimination_1965,stuhrmann_elimination_1967}) for monodisperse particles. The so-called \emph{basic functions} formulation differentiates, independently of the particle inner structure, the contributions which depend on the varying solvent density or contrast (\(\Delta\eta\)) and on the excess of electron density of each component of the particle. 

Deriving from this approach, the scattering intensity can be expressed as the combination of contributions corresponding to different features of the particles:
\begin{equation}
\label{eq:intensity_contrast}
I(q)=I_c(q)+\Delta\eta I_{sc}(q)+(\Delta\eta)^2 I_{s}(q)
\end{equation}
The $I_c$ function contains the contributions from the density fluctuations inside the particle, the contribution $I_s$ is the so-called \emph{resonant term} and $I_{sc}$ is the cross-term function.


\subsubsection{Shape factor}
The $I_s(q)$ function, also known as \emph{shape factor}, corresponds to the scattering contributions from particles with homogenous density and a size equivalent to the volume inaccessible to the solvent. By modelling the shape factor function, the shape and size distribution of the polymeric colloids can be determined independently of their inner structure.

For this purpose, a spherical form factor for homogeneous colloids with a gaussian size distribution was utilized, similarly to the PMMA-COOH example. In order to obtain the particle sphericity, an ellipsoid model was employed.


\subsubsection{Guinier law}
\label{sec:TheoryGuinier}
Gyration radius

The radius of gyration \( R_g\) is systematically employed in small-angle scattering as an evaluation tool \cite{mertens_structural_2010,sim_salt_2012}. It can be calculated using the Guinier approximation \cite{guinier_diffraction_1939,guinier_small-angle_1955}, which assumes that the scattering intensity behaves in the limit of small \(q\) as
\begin{equation}
\label{eq:guinier}
I(q)=I(0)\,\mbox{exp}\left(-\frac{R_g^2}{3}q^2\right),
\end{equation}
where \( I(0)\) is known as forward scattering or intensity at zero angle. Using the basic functions approach, the radius of gyration of a monodisperse, heterogeneous particle can be expressed as a function of the solvent electron density \( \rho_{solv} \) and the average electron density of the particle \( \rho_0 \) \cite{feigin_structure_1987}
\begin{equation}
R_g^2=R_{g,c}^{\,2}+\frac{\alpha}{\rho_0-\rho_{solv}}-\frac{\beta}{(\rho_0-\rho_{solv})^2},
\label{eq:gyration}
\end{equation}
where \(R_{g,c}\) is the radius of gyration of the particle shape corresponding to the volume inaccessible for the solvent \( V_c \), \( \alpha \) characterizes the distribution of different phases inside the particle and \( \beta>0 \) considers the eccentricity of the different scattering contributions \cite{stuhrmann_small-angle_2008}. Nevertheless, particle aggregation influences the scattering curves especially in the Guinier region and must be explicitly avoided.

\cite{avdeev_contrast_2007} proposed an extended version to equation \eqref{eq:gyration} for the case of a polydisperse particle ensemble by introducing the \emph{effective} values \( \tilde R^2_{g,c} \), \( \tilde \alpha \) and \( \tilde \beta \), which are the intensity-weighted averages of the corresponding parameters over the polydispersity. The observed average electron density is not affected by the polydispersity (\( \tilde\rho_0=\rho_0 \)) if the volume ratio between the different particle components is constant for all particles in the ensemble.

Assuming the same premise, the intensity at zero angle is given by
\begin{equation}
\label{eq:I0}
I(0)\propto N \left( \rho_0-\rho_{solv} \right)^2 ,
\end{equation}
with a minimum at \( \rho_{solv}=\rho_0 \). Therefore, by analyzing the Guinier region of the scattering curves, the average electron density of the particle can be obtained without assuming an \emph{a priori} inner structure.

Using the models presented above, it is possible to obtain by independent means the external radius and the average electron density of the particle in suspension.
\subsubsection{$I(0)$}
what happens in polydisperse systems?

\section{Dynamic Light Scattering}

The technique was used extensively in this thesis.
