\chapter*{Abstract}
\thispagestyle{empty}


In the continuously growing field of nanomedicine, nanonoparticles have a preeminent position, opening exciting new possibilities as platforms for drug-delivery or encapsulating imaging agents. Indeed, polymeric colloids are starting to undergo clinical trials and a lipid vesicle was used as nanocarrier for the first approved nano-drug, Doxil. Therefore, the current advances in nanomaterial development are focused towards tailoring polymeric nano-drug carriers with flexible surface functionalization and controlled morphologies, defining aspects of the particle functions e.g. their \emph{in vivo} biodistribution or their drug-delivery efficacy. 

However, most current characterization techniques possess certain limitations i.e. cannot prove the innner structure present in many low-density nanoparticles. This work proposes a novel approach to contrast variation with SAXS \textbf{[1]} based on the constitution of a solvent density gradient in a glass capillary in order to choose \emph{in situ} the most appropriate contrast and to acquire extensive datasets in a short time interval.

By examining the scattering curves measured at different aqueous sucrose densities, information about the internal morphology of the nanoparticles as well as their size distribution can be obtained. Additionally an estimation of the particle density can be determined focusing on the Guinier region of the curve, as shown for polymeric colloids across a wide spectrum of polymers \textbf{[2]}. These results were successfully compared with techniques such as DCS and several imaging methods.

The continuous contrast variation technique was also employed to characterize the nano-drug Doxil$\textregistered$, a PEGylated liposomal formulation of doxorubicin, using iodixanol as contrast agent, an iso-osmolar suspending medium. The study is focused on the isoscattering point position and the model-free analysis of the scattering curves and highlights the advantages in comparison to widespread characterization techniques as DLS and TEM \textbf{[3]}.

Furthermore, the response of the nanocarrier to increasing solvent osmolality is evaluated with sucrose contrast variation and compared to the different response of PEGylated and plain liposomes to osmotic pressure depending on their size. The osmotic pressure needed for the liposomal shrinkage is quantitatively studied by focusing on the evolution of the isoscattering point intensity, while the study of the phospholipid bilayer scattering feature gives an insight into the morphological changes induced by the osmotic shrinkage.

The possibilities as sizing technique of the continuous contrast variation method are further investigated on relevant bio-materials like human lipoproteins or polymeric nanocarriers coated with antibodies. In addition, this technique is employed to determine the density of the lipoproteins, one of the most characteristic traits of these blood plasma components.

\bigskip
\footnotesize{

\textbf{[1]} R. Garcia-Diez, C. Gollwitzer, M. Krumrey, \emph{J. Appl. Cryst.} \textbf{48}, 20-28 (2015)

\textbf{[2]} R. Garcia-Diez, A. Sikora, C. Gollwitzer, C. Minelli, M. Krumrey, \emph{Eur. Polym. J.} \textbf{81}, 641-649 (2016)

\textbf{[3]} R. Garcia-Diez, C. Gollwitzer, M. Krumrey, Z. Varga, \emph{Langmuir} \textbf{32 (3)}, 772-778 (2015)

}
\normalsize

\cleardoublepage

\thispagestyle{empty}
\selectlanguage{ngerman}

\chapter*{Zusammenfassung}

\textcolor{red}{A GOOD TRANSLATION WILL BE DONE ONCE THE ENGLISH ABSTRACT IS DEFINITIVE! THAT WAS ONLY GOOGLE TRANSLATOR!!!}Im kontinuierlich wachsenden Bereich der Nanomedizin haben Nanopartikel eine herausragende Stellung und eröffnen aufregende neue Möglichkeiten als Plattform für Wirkstoffabgabe oder Verkapselung bildgebender Mittel. Tatsächlich beginnen polymere Kolloide, klinische Versuche durchzuführen, und ein Lipidvesikel wurde als Nanoträger für das erste zugelassene Nano-Medikament (Doxil) verwendet. Daher sind die gegenwärtigen Fortschritte in der Entwicklung von Nanomaterialien darauf ausgerichtet, polymere Nano-Arzneimittelträger mit flexibler Oberflächenfunktionalisierung und gesteuerten Morphologien herzustellen, wobei Aspekte der Teilchenfunktionen, z.B. Ihre \emph{in vivo}-Bioverteilung oder ihre Wirksamkeit bei der Wirkstoffabgabe.

Jedoch besitzen die meisten gegenwärtigen Charakterisierungstechniken gewisse Beschränkungen, d.h. nicht die innere Struktur beweisen, die in vielen Nanopartikeln niedriger Dichte vorhanden ist. Diese Arbeit schlägt einen neuen Ansatz zur Kontrastveränderung mit SAXS auf der Grundlage des Aufbaus eines Lösungsmitteldichtegradienten in einer Glaskapillare vor \textbf{[1]}, um \emph{in situ} den geeignetsten Kontrast zu wählen und umfangreiche Datensätze in einem kurzen Zeitintervall zu erwerben.

Durch Untersuchung der bei unterschiedlichen wässrigen Sucrose-Dichten gemessenen Streukurven können Informationen über die innere Morphologie der Nanopartikel sowie deren Größenverteilung erhalten werden. Zusätzlich kann eine Schätzung der Partikeldichte mit Fokussierung auf den Guinier-Bereich der Kurve, wie für polymere Kolloide über ein breites Spektrum von Polymeren gezeigt, bestimmt werden \textbf{[2]}. Diese Ergebnisse wurden erfolgreich mit Techniken wie DCS und mehrere bildgebende Verfahren verglichen.

Die kontinuierliche Kontrastveränderungstechnik wurde ebenfalls verwendet, um Doxil, eine PEGylierte liposomale Formulierung von Doxorubicin, unter Verwendung von Iodixanol als Kontrastmittel, eines iso-osmolaren Suspensionsmediums, zu charakterisieren. Die Studie konzentriert sich auf die Isoscattering-Point-Position und die modellfreie Analyse der Streukurven und hebt die Vorteile im Vergleich zu weit verbreiteten Charakterisierungstechniken als DLS und TEM hervor \textbf{[3]}.

Darüber hinaus wird die Reaktion des Nanoträgers auf die zunehmende Lösungsmittel-Osmolalität mit einer Saccharose-Kontrast-Variation bewertet und mit der unterschiedlichen Reaktion von PEGylierten und einfachen Liposomen auf den osmotischen Druck in Abhängigkeit von ihrer Größe verglichen. Der für die liposomale Schrumpfung benötigte osmotische Druck wird quantitativ durch die Fokussierung auf die Evolution der Isoscattering-Punkt-Intensität untersucht, während die Untersuchung der Phospholipid-Doppelschicht-Streuung einen Einblick in die morphologischen Veränderungen der osmotischen Schrumpfung gibt.

Die Möglichkeiten der Dimensionierung des kontinuierlichen Kontrastvariationsverfahrens werden auch auf relevanten Biomaterialien wie menschlichen Lipoproteinen oder mit Antikörpern beschichteten polymeren Nanocarrieren gezeigt. Zusätzlich wird diese Technik verwendet, um die Dichte der Lipoproteine zu bestimmen, eine der charakteristischsten Merkmale dieser Blutplasmakomponenten.

\bigskip
\footnotesize{

\textbf{[1]} R. Garcia-Diez, C. Gollwitzer, M. Krumrey, \emph{J. Appl. Cryst.} \textbf{48}, 20-28 (2015)

\textbf{[2]} R. Garcia-Diez, A. Sikora, C. Gollwitzer, C. Minelli, M. Krumrey, \emph{Eur. Polym. J.} \textbf{81}, 641-649 (2016)

\textbf{[3]} R. Garcia-Diez, C. Gollwitzer, M. Krumrey, Z. Varga, \emph{Langmuir} \textbf{32 (3)}, 772-778 (2015)

}
\normalsize

\selectlanguage{UKenglish}
\cleardoublepage
