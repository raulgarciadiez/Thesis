\chapter*{Abstract}
\thispagestyle{empty}


In the continuously growing field of nanomedicine, nanonoparticles have a pre-eminent position, opening exciting new possibilities as platforms for drug-delivery or encapsulating imaging agents. Indeed, a lipid vesicle was used as a nanocarrier for the first approved nano-drug, Doxil\textregistered\ (Caelyx\textregistered\ in Europe), and polymeric colloids are starting to undergo clinical trials. Therefore, the current advances in nanomaterial development are focused towards tailoring nano-drug carriers with flexible surface functionalization and controlled morphologies, defining aspects of the particle functions e.g. their \emph{in vivo} biodistribution or their drug-delivery efficacy. 

However, most current characterization techniques possess certain limitations i.e. cannot prove the inner structure present in lipid vesicles and many low-density nanoparticles. This work proposes a novel approach to contrast variation in small-angle X-ray scattering (SAXS) based on the constitution of a solvent density gradient in a glass capillary in order to choose \emph{in situ} the most appropriate contrast and to acquire extensive datasets in a short time interval.

%However, most current characterization techniques possess certain limitations i.e. cannot prove the inner structure present in lipid vesicles and many low-density nanoparticles. This work proposes a novel approach to contrast variation in small-angle X-ray scattering (SAXS) \textbf{[1]} based on the constitution of a solvent density gradient in a glass capillary in order to choose \emph{in situ} the most appropriate contrast and to acquire extensive datasets in a short time interval.

By examining the scattering curves measured at different aqueous sucrose densities, information about the internal morphology of the nanoparticles as well as their size distribution can be obtained. Additionally, an estimation of the particle density can be determined focusing on the Guinier region of the curve, as shown for polymeric colloids across a wide spectrum of polymers. These results were successfully compared with several imaging methods and techniques such as Differential Centrifugal Sedimentation.
 
%By examining the scattering curves measured at different aqueous sucrose densities, information about the internal morphology of the nanoparticles as well as their size distribution can be obtained. Additionally an estimation of the particle density can be determined focusing on the Guinier region of the curve, as shown for polymeric colloids across a wide spectrum of polymers \textbf{[2]}. These results were successfully compared with techniques such as Differential Centrifugal Sedimentation (DCS) and several imaging methods.

The continuous contrast variation technique was also employed to characterize the nano-drug Caelyx, a PEGylated liposomal formulation of doxorubicin, using iodixanol as contrast agent, an iso-osmolar suspending medium. The study is focused on the isoscattering point position and the model-free analysis of the scattering curves and highlights the advantages in comparison to widespread characterization techniques as Dynamic Light Scattering and Transmission Electron Microscopy.

%The continuous contrast variation technique was also employed to characterize the nano-drug Caelyx, a PEGylated liposomal formulation of doxorubicin, using iodixanol as contrast agent, an iso-osmolar suspending medium. The study is focused on the isoscattering point position and the model-free analysis of the scattering curves and highlights the advantages in comparison to widespread characterization techniques as Dynamic Light Scattering (DLS) and Transmission Electron Microscopy (TEM) \textbf{[3]}.

Furthermore, the response of the nanocarrier to increasing solvent osmolality is evaluated with sucrose contrast variation and compared to the different response of PEGylated and plain liposomes to osmotic pressure depending on their size. The osmotic pressure needed for the liposomal shrinkage is quantitatively studied by focusing on the evolution of the isoscattering point intensity, while the study of the phospholipid bilayer scattering feature gives an insight into the morphological changes induced by the osmotic shrinkage.

The possibilities as a sizing technique of the continuous contrast variation method are further investigated on relevant bio-materials like human lipoproteins or polymeric nanocarriers coated with antibodies. In addition, this technique is employed to determine the density of the lipoproteins, one of the most characteristic traits of these blood plasma components.

%\bigskip
%\footnotesize{

%\textbf{[1]} R. Garcia-Diez, C. Gollwitzer, M. Krumrey, \emph{J. Appl. Cryst.} \textbf{48}, 20-28 (2015)

%\textbf{[2]} R. Garcia-Diez, A. Sikora, C. Gollwitzer, C. Minelli, M. Krumrey, \emph{Eur. Polym. J.} \textbf{81}, 641-649 (2016)

%\textbf{[3]} R. Garcia-Diez, C. Gollwitzer, M. Krumrey, Z. Varga, \emph{Langmuir} \textbf{32 (3)}, 772-778 (2015)

%}
\normalsize

\cleardoublepage
%\clearpage

\thispagestyle{empty}
\selectlanguage{ngerman}

\chapter*{Zusammenfassung}

{\fontsize{10}{11}\selectfont
%\renewcommand{\baselinestretch}{2}
\linespread{1.6}
\setlength{\parindent}{1em}
\setlength{\parskip}{0.5em}
Im kontinuierlich wachsenden Bereich der Nanomedizin haben Nanopartikel eine herausragende Stellung und eröffnen aufregende neue Möglichkeiten als Plattform für Wirkstofftransport oder Verkapselung von Kontrastmitteln. Tatsächlich wurde ein Lipidvesikel verwendet als Nanocarrier für das erste zugelassene Nano-Arzneimittel, Doxil\textregistered\ (Caelyx\textregistered\ in Europe), und polymere Nanopartikel beginnen klinische Prüfungen durchzuführen. Daher sind die aktuelle Fortschritte in der Entwicklung von Nanomaterialien darauf ausgerichtet, Nanocarriers mit flexibler Oberflächenfunktionalisierung und kontrollierten Morphologien herzustellen, definierende Merkmale der Teilchenfunktionen, z.B. ihre \emph{in vivo} Bioverteilung oder ihre Wirksamkeit bei der Wirkstofffreisetzung.

Jedoch besitzen die meisten gegenwärtigen Charakterisierungstechniken gewisse Beschränkungen, d.h. die innere Struktur, die in Lipidvesikeln und vielen Nanopartikeln niedriger Dichte vorliegt, kann nicht untersucht werden. Diese Arbeit schlägt einen neuartigen Ansatz zur Kontrastvariation in Röntgen-Kleinwinkel-Streuung (SAXS) auf der Grundlage des Aufbaus eines Lösungsmitteldichtegradienten in einer Glaskapillare vor, um \emph{in situ} den geeignetsten Kontrast zu wählen und umfangreiche Datensätze in einer kurze Zeitraum zu sammeln.

Informationen über die innere Morphologie der Nanopartikel sowie deren Größenverteilung können durch Untersuchung der Streukurven, die bei verschiedenen Konzentrationen von Zucker in Wasser gemessen werden, erhalten werden. Zusätzlich kann die Teilchendichte bestimmt werden, wenn man sich auf den Guinier-Bereich der Streukurven fokussiert, wie für polymere Nanopartikel über ein breites Spektrum von Teilchendichten gezeigt ist. Diese Ergebnisse wurden erfolgreich verglichen mit mehreren bildgebenden Verfahren und Techniken wie differenziell-zentrifugale Sedimentation (DCS).

Die kontinuierliche Kontrastvariation-Technik wurde ebenfalls mit Iodixanol als Kontrastmittel, eines iso-osmolaren Suspensionsmedium, verwendet, um das Nano-Arzneimittel Caelyx, eine PEGylierte liposomale Formulierung von Doxorubicin, zu charakterisieren. Die Studie konzentriert sich auf die Position der \emph{Isoscatteringpunkte} und die modellfreie Analyse der Streukurven und unterstreicht die Vorteile im Vergleich zu weit verbreiteten Charakterisierungstechniken wie dynamische Lichtstreuung (DLS) und Transmissionselektronenmikroskop (TEM).

Weiterhin wird die Reaktion des Nanocarriers auf eine zunehmende Lösungsmittel-Osmolalität mittels Zucker-Kontrastvariation bewertet und mit der unterschiedlichen Reaktion von PEGylierten und einfachen Liposomen auf den osmotischen Druck in Abhängigkeit von ihrer Größe verglichen. Der osmotische Druck benötigt für die liposomale Schrumpfung wird quantitativ untersucht mittels der Evolution der Isoscatteringpunkt-Intensität, während die Untersuchung der Streuung der Phospholipid-Doppelschicht einen Einblick in die morphologischen Veränderungen der osmotischen Schrumpfung gibt.

Die Möglichkeiten des kontinuierlichen Kontrastvariation-Methode als Technik für Grössen-Bestimmung werden weiter auf relevanten Biomaterialien wie menschlichen Lipoproteinen oder polymeren Nanocarriers, die mit Antikörpern beschichtet sind, untersucht. Zusätzlich wird diese Technik verwendet, um die Dichte der Lipoproteine zu bestimmen, eine der Haupteigenschaften dieser Blutplasmakomponenten.

}



%\bigskip
%\footnotesize{

%\textbf{[1]} R. Garcia-Diez, C. Gollwitzer, M. Krumrey, \emph{J. Appl. Cryst.} \textbf{48}, 20-28 (2015)

%\textbf{[2]} R. Garcia-Diez, A. Sikora, C. Gollwitzer, C. Minelli, M. Krumrey, \emph{Eur. Polym. J.} \textbf{81}, 641-649 (2016)

%\textbf{[3]} R. Garcia-Diez, C. Gollwitzer, M. Krumrey, Z. Varga, \emph{Langmuir} \textbf{32 (3)}, 772-778 (2015)

%}
\normalsize

\selectlanguage{UKenglish}
\cleardoublepage
