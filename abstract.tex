%\chapter*{Abstract}


\thispagestyle{plain}
\begin{center}
    \Large
    \textbf{Characterization of nanoparticles by continuous contrast variation in SAXS}
    
    \vspace{0.4cm}
    \large
    Physikalische-Technische Bundesanstalt
    
    \vspace{0.4cm}
    \textbf{Raul Garcia-Diez}
    
    \vspace{0.9cm}
    \textbf{Abstract}
\end{center}

In the continuously growing field of nanomedicine, nanonoparticles have a preeminent position, opening exciting new possibilities as platforms for drug-delivery or encapsulating imaging agents. Indeed, polymeric colloids are starting to undergo clinical trials and a lipid vesicle was used as nanocarrier for the first approved nano-drug, Doxil$\textregistered$. Therefore, the current advances in nanomaterial development are focused towards tailoring polymeric nano-drug carriers with flexible surface functionalisation and controlled morphologies, defining aspects of the particle functions e.g. their \emph{in vivo} biodistribution or their drug-delivery efficacy. 

However, most current characterization techniques possess certain limitations i.e. cannot prove the innner structure present in many low-density nanoparticles. This work proposes a novel approach to contrast variation with SAXS \textbf{[1]} based on the constitution of a solvent density gradient in a glass capillary in order to choose \emph{in situ} the most appropriate contrast and to acquire extensive datasets in a short time interval.

By examining the scattering curves measured at different aqueous sucrose densities, information about the internal morphology of the nanoparticles as well as their size distribution can be obtained. Additionally an estimation of the particle density can be determined focusing on the Guinier region of the curve, as shown for polymeric colloids across a wide spectrum of polymers \textbf{[2]}. These results were successfully compared with techniques such as DCS and several imaging methods.

The continuous contrast variation technique was also employed to characterize Doxil$\textregistered$, a PEGylated liposomal formulation of doxorubicin, using iodixanol as contrast agent, an iso-osmolar suspending medium. The study is focused on the isoscattering point position and the model-free analysis of the scattering curves and highlights the advantadges in comparison to widely used characterization techniques as DLS and TEM \textbf{[3]}. 

Furthermore, the response of the nanocarrier to increasing solvent osmolality is evaluated with sucrose contrast variation and compared to the different response of PEGylated and plain liposomes to osmotic pressure depending on their size. For instance, the osmotic pressure needed for the liposomal shrinkage is quantitatively studied by focusing on the evolution of the isoscattering point intensity, which gives an insight into the Laplace law for small sized sterically stabilized liposomes and the role of the PEG moieties in the membrane resilience.

\bigskip

\footnotesize{

\textbf{[1]} R. Garcia-Diez, C. Gollwitzer, M. Krumrey, \emph{J. Appl. Cryst.} \textbf{48}, 20-28 (2015)

\textbf{[2]} R. Garcia-Diez, A. Sikora, C. Gollwitzer, C. Minelli, M. Krumrey, \emph{Eur. Polym. J.} (2016)

\textbf{[3]} R. Garcia-Diez, C. Gollwitzer, M. Krumrey, Z. Varga, \emph{Langmuir} \textbf{32 (3)}, 772-778 (2015)

}

\normalsize

