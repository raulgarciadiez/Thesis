\chapter*{Abstract}
\thispagestyle{empty}

%opening exciting new possibilities as platforms for drug-delivery or encapsulating imaging agents. Indeed, a lipid vesicle was used as a nanocarrier for the first approved nano-drug, Doxil\textregistered\ (Caelyx\textregistered\ in Europe), and polymeric colloids are starting to undergo clinical trials. Therefore, the current advances in nanomaterial development are focused towards tailoring nano-drug carriers with flexible surface functionalization and controlled morphologies, defining aspects of the particle functions e.g. their \emph{in vivo} biodistribution or their drug-delivery efficacy. 

%However, most current characterization techniques possess certain limitations i.e. cannot prove the inner structure present in lipid vesicles and many low-density nanoparticles. 

In the continuously growing field of nanomedicine, nanoparticles have a pre-eminent position. The particle morphology is a defining aspect of their functionality, yet most current characterization techniques possess certain limitations. This work proposes a novel approach to contrast variation in small-angle X-ray scattering based on the constitution of a solvent density gradient in a glass capillary in order to choose \emph{in situ} the most appropriate contrast and to acquire extensive datasets in a short time interval.

%However, most current characterization techniques possess certain limitations i.e. cannot prove the inner structure present in lipid vesicles and many low-density nanoparticles. This work proposes a novel approach to contrast variation in small-angle X-ray scattering (SAXS) \textbf{[1]} based on the constitution of a solvent density gradient in a glass capillary in order to choose \emph{in situ} the most appropriate contrast and to acquire extensive datasets in a short time interval.

By examining the scattering curves measured at different aqueous sucrose concentrations, information about the internal structure of the nanoparticles as well as their size distribution is obtained. Additionally, the particle density can be estimated from the Guinier region of the scattering curve, as is shown for polymeric colloids across a wide spectrum of polymers. These results are successfully compared with imaging methods and other techniques such as Differential Centrifugal Sedimentation.

% Of special interest is the position of the so-called isoscattering point, which is the intersection in a single point of all the scattering curves at different contrast and is directly related with the mean particle size. 
 
%By examining the scattering curves measured at different aqueous sucrose densities, information about the internal morphology of the nanoparticles as well as their size distribution can be obtained. Additionally an estimation of the particle density can be determined focusing on the Guinier region of the curve, as shown for polymeric colloids across a wide spectrum of polymers \textbf{[2]}. These results were successfully compared with techniques such as Differential Centrifugal Sedimentation (DCS) and several imaging methods.

The continuous contrast variation technique is also employed to characterize the nano-drug Caelyx, a PEGylated liposomal formulation of doxorubicin, using iodixanol as contrast agent, an iso-osmolar suspending medium. The mean size of the nanocarrier is obtained by a model-free analysis of the scattering curves based on the position of the so-called \emph{isoscattering point}, while the traceable determination of the particle size highlights the advantages in comparison to widespread characterization techniques as Dynamic Light Scattering and Transmission Electron Microscopy.

%A model-free analysis of the scattering curves based on the position of the so-called isoscattering point is performed jointly with a traceable determination of the particle size distribution which highlights the advantages in comparison to widespread characterization techniques as Dynamic Light Scattering and Transmission Electron Microscopy.

%By means of the so-called isoscattering point position, a model-free analysis of the scattering curves is performed and highlights the advantages in comparison to widespread characterization techniques as Dynamic Light Scattering and Transmission Electron Microscopy.

%The continuous contrast variation technique was also employed to characterize the nano-drug Caelyx, a PEGylated liposomal formulation of doxorubicin, using iodixanol as contrast agent, an iso-osmolar suspending medium. The study is focused on the isoscattering point position and the model-free analysis of the scattering curves and highlights the advantages in comparison to widespread characterization techniques as Dynamic Light Scattering (DLS) and Transmission Electron Microscopy (TEM) \textbf{[3]}.

Furthermore, the response of the nanocarrier to increasing solvent osmolality is evaluated with sucrose contrast variation and compared to the different response of PEGylated and plain liposomes to osmotic pressure depending on their size. Therefore, the osmotic pressure necessary for the liposomal shrinkage is quantitatively studied and the morphological changes induced by this deformation are thoroughly examined.

% by focusing on the evolution of the isoscattering point intensity, while the study of the phospholipid bilayer scattering feature gives an insight into the morphological changes induced by the osmotic shrinkage.

The capabilities of the continuous contrast variation method as a sizing technique  are further investigated on relevant bio-materials like human lipoproteins or polymeric nanocarriers coated with antibodies. In addition, this technique is employed to determine the density of the lipoproteins, one of the most characteristic traits of these blood plasma components.

%\bigskip
%\footnotesize{

%\textbf{[1]} R. Garcia-Diez, C. Gollwitzer, M. Krumrey, \emph{J. Appl. Cryst.} \textbf{48}, 20-28 (2015)

%\textbf{[2]} R. Garcia-Diez, A. Sikora, C. Gollwitzer, C. Minelli, M. Krumrey, \emph{Eur. Polym. J.} \textbf{81}, 641-649 (2016)

%\textbf{[3]} R. Garcia-Diez, C. Gollwitzer, M. Krumrey, Z. Varga, \emph{Langmuir} \textbf{32 (3)}, 772-778 (2015)

%}
\normalsize

\cleardoublepage
%\clearpage

\thispagestyle{empty}
\selectlanguage{ngerman}

\chapter*{Zusammenfassung}

%{\fontsize{10}{11}\selectfont
%%%%%%%%%%%%\renewcommand{\baselinestretch}{2}
%\linespread{1.6}
%\setlength{\parindent}{1em}
%\setlength{\parskip}{0.5em}

Im kontinuierlich wachsenden Bereich der Nanomedizin haben Nanopartikel eine herausragende Stellung. Die funktionalen Eigenschaften der Nanopartikeln werden durch ihre Morphologie beeinflusst, jedoch haben die meisten gegenwärtigen Charakterisierungstechniken gewisse Einschränkungen. Die vorliegende Arbeit schlägt einen neuartigen Ansatz zur Kontrastvariation in Röntgen-Kleinwinkel-Streuung (SAXS, \emph{Small-Angle X-ray Scattering}) auf der Grundlage des Aufbaus eines Lösungsmitteldichtegradienten in einer Glaskapillare vor, um \emph{in situ} den geeignetsten Kontrast zu wählen und umfangreiche Datensätze innerhalb eines kurzen Zeitraums zu sammeln.

%kontrollierten Morphologien herzustellen, definierende Merkmale der Teilchenfunktionen / funtktionale Eigenschaften zu beeinflussen

% und eröffnen aufregende neue Möglichkeiten als Plattform für Wirkstofftransport oder Verkapselung von Kontrastmitteln. Tatsächlich wurde ein Lipidvesikel verwendet als Nanocarrier für das erste zugelassene Nano-Arzneimittel, Doxil\textregistered\ (Caelyx\textregistered\ in Europe), und polymere Nanopartikel beginnen klinische Prüfungen durchzuführen. Daher sind die aktuelle Fortschritte in der Entwicklung von Nanomaterialien darauf ausgerichtet, Nanocarriers mit flexibler Oberflächenfunktionalisierung und kontrollierten Morphologien herzustellen, definierende Merkmale der Teilchenfunktionen, z.B. ihre \emph{in vivo} Bioverteilung oder ihre Wirksamkeit bei der Wirkstofffreisetzung.

%Jedoch besitzen die meisten gegenwärtigen Charakterisierungstechniken gewisse Beschränkungen, d.h. die innere Struktur, die in Lipidvesikeln und vielen Nanopartikeln niedriger Dichte vorliegt, kann nicht untersucht werden. Diese Arbeit schlägt einen neuartigen Ansatz zur Kontrastvariation in Röntgen-Kleinwinkel-Streuung (SAXS) auf der Grundlage des Aufbaus eines Lösungsmitteldichtegradienten in einer Glaskapillare vor, um \emph{in situ} den geeignetsten Kontrast zu wählen und umfangreiche Datensätze in einer kurze Zeitraum zu sammeln.

Informationen über die innere Struktur von Nanopartikeln sowie deren Größenverteilung können durch Untersuchung der Streukurven, die bei verschiedenen Konzentrationen von Zucker in Wasser gemessen werden, erhalten werden. Zusätzlich kann die Teilchendichte bestimmt werden, indem der Guinier-Bereich der Streukurven analysiert wird, was für polymere Nanopartikel über ein breites Spektrum von Teilchendichten gezeigt wird. Diese Ergebnisse wurden erfolgreich mit mikroskopischen und anderen Techniken wie Sedimentation in einem Dichtegradient (DCS, Differential Centrifugal Sedimentation) verglichen.

Die Technik der kontinuierlichen Kontrastvariation wurde mit dem iso-osmolaren Kontrastmittel Iodixanol auch an dem Nano-Arzneimittel Caelyx durchgeführt, einer PEGylierten liposomalen Zubereitung des Medikaments Doxorubicin. Die mittlere Größe des Nanocarriers wird durch eine modellfreie Analyse der Streukurven basierend auf der Position der sogenannten \emph{Isoscattering-Punkte} erhalten, während die rückführbare Bestimmung der Partikelgrößen die Vorteile im Vergleich zu weit verbreiteten Charakterisierungstechniken wie dynamischer Lichtstreuung (DLS, Dynamic Light Scattering) und Transmissionselektronenmikroskopie (TEM) unterstreicht.


%Mit Hilfe der so genannten \emph{Isoscatteringpunkte} wird eine weitgehend modellfreie Analyse der Streukurven durchgeführt und unterstreicht die Vorteile im Vergleich zu weit verbreiteten Charakterisierungstechniken wie dynamische Lichtstreuung (DLS) und Transmissionselektronenmikroskop (TEM).


Zusätzlich wird die Reaktion des Nanocarriers auf eine zunehmende Lösungsmittel-Osmolalität mittels Zucker-Konzentrationsvariation untersucht und die unterschiedlichen Reaktionen von PEGylierten und einfachen Liposomen auf den osmotischen Druck in Abhängigkeit ihrer Größe verglichen. Dafür wird der für die liposomale Schrumpfung benötigte osmotische Druck quantitativ analysiert und die durch diese Deformation induzierten morphologischen Veränderungen sorgfältig untersucht.


%Der osmotische Druck benötigt für die liposomale Schrumpfung wird quantitativ untersucht mittels der Evolution der Isoscatteringpunkt-Intensität, während die Untersuchung der Streuung der Phospholipid-Doppelschicht einen Einblick in die morphologischen Veränderungen der osmotischen Schrumpfung gibt.


Die Möglichkeiten der kontinuierlichen Kontrastvariationmethode als Technik zur Grössenbestimmung werden weiter anhand von relevanten Biomaterialien untersucht, wie menschlichen Lipoproteinen oder polymeren Nanocarriern, die mit Antikörpern beschichtet sind. Außerdem wird diese Technik verwendet, um die Dichte von Lipoproteinen zu bestimmen, eine der Haupteigenschaften dieser Blutplasmakomponenten.

%}



%\bigskip
%\footnotesize{

%\textbf{[1]} R. Garcia-Diez, C. Gollwitzer, M. Krumrey, \emph{J. Appl. Cryst.} \textbf{48}, 20-28 (2015)

%\textbf{[2]} R. Garcia-Diez, A. Sikora, C. Gollwitzer, C. Minelli, M. Krumrey, \emph{Eur. Polym. J.} \textbf{81}, 641-649 (2016)

%\textbf{[3]} R. Garcia-Diez, C. Gollwitzer, M. Krumrey, Z. Varga, \emph{Langmuir} \textbf{32 (3)}, 772-778 (2015)

%}
\normalsize

\selectlanguage{UKenglish}
\cleardoublepage
