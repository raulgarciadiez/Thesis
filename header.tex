%%%% common header file for final and draft mode
\usepackage[utf8]{inputenc}
\usepackage[T1]{fontenc}
%\usepackage{lmodern}
\usepackage[ngerman,UKenglish]{babel}
\usepackage{microtype}
\usepackage{textcomp}
\usepackage{xspace}
\usepackage[dvipsnames,svgnames]{xcolor}
\usepackage{floatrow}
\DeclareColorBox{imgbg}{\fcolorbox{white}{white}}

%%%% Textsatz, Layout, Stil
\usepackage{mathpazo}

\linespread{1.05}         % Palatino needs more leading (space between lines)
\usepackage[scaled=1.03]{inconsolata}

%Überschriften serifenlos und über den Rand hängend
\usepackage[sf,sl,outermarks,noindentafter,nobottomtitles]{titlesec}

%könnte alles auch mit \bfseries versehen werden nach Geschmack
\titleformat{\section}[hang]{\LARGE\rmfamily}{\thetitle}{8pt}{}
\titleformat{\subsection}[hang]{\Large\rmfamily}{\thetitle}{8pt}{}
\titleformat{\subsubsection}[hang]{\bfseries\large\rmfamily}{\thetitle}{8pt}{}
\titleformat{\paragraph}[hang]{\bfseries\rmfamily}{\thetitle}{8pt}{}

%Etwas aufwendiger für die Kapitelüberschriften:
\titleformat{\chapter}[display]
{\filleft\Huge\rmfamily} %Huge ist die Größe für Titeltext und Nummer
{\fontsize{100pt}{90pt}\selectfont\thechapter}
{-2ex} %is vertical space in [display] mode
%Platz vor dem ganzen Krempel
{\vspace{1ex}}
%Platz danach
[\vspace{1ex}]

%%% TOC design
\usepackage[titles]{tocloft}
\setlength{\cftbeforechapskip}{1ex}
\setlength{\cftbeforesecskip}{0.8ex}
\setlength{\cftbeforesubsecskip}{0.8ex}

%\renewcommand{\cftchapfont}{\sffamily\bfseries}
%\renewcommand{\cftchappagefont}{\sffamily}
%\renewcommand{\cftsecfont}{\sffamily}
%\renewcommand{\cftsecpagefont}{\sffamily}
%\renewcommand{\cftsubsecfont}{\sffamily}
%\renewcommand{\cftsubsecpagefont}{\sffamily}

%\renewcommand{\cftpnumalign}{r}
%\renewcommand{\cftsecdotsep}{\cftnodots}
%\renewcommand{\cftsubsecdotsep}{\cftnodots}
%\renewcommand{\cftchapleader}{\hspace{2em}}
%\renewcommand{\cftsecleader}{\hspace{2em}}
%\renewcommand{\cftsubsecleader}{\hspace{2em}}
%\renewcommand{\cftchapafterpnum}{\cftparfillskip}
%\renewcommand{\cftsecafterpnum}{\cftparfillskip}
%\renewcommand{\cftsubsecafterpnum}{\cftparfillskip}



%%%% Grafiken, Abbildungen
\floatsetup{
    capposition=beside,
    capbesideposition={center,outside},
%    capbesideposition={top,outside},    
    facing=yes, 
    floatwidth=.75\linewidth,
    capbesidewidth=sidefil,
    capbesidesep=quad,
    floatrowsep=quad,
    %framestyle=colorbox,framearound=all,colorframeset=imgbg,frameset={\fboxrule0pt},
    }
\floatsetup[widefigure]{%
    floatwidth=0.95\textwidth,
    %margins=hangoutside,
    %capposition=beside,
    capposition=below,
    %capbesideposition={top,outside},
    %capbesidewidth=\marginparwidth,
    %capbesideframe=yes,
    %capbesidesep=columnsep,
    %floatrowsep=columnsep,
    %heightadjust=nocaption,
    facing=yes,
    }
\floatsetup[widetable]{%
    floatwidth=0.95\textwidth,
    %margins=hangoutside,
    %capposition=beside,
    capposition=above,
    %capbesideposition={top,outside},
    %capbesidewidth=\marginparwidth,
    %capbesideframe=yes,
    %capbesidesep=columnsep,
    %floatrowsep=columnsep,
    %heightadjust=nocaption,
    facing=yes,
    }
\floatsetup[widefloat]{%
    floatwidth=0.95\textwidth,
    %margins=hangoutside,
    %capposition=beside,
    capposition=below,
    %capbesideposition={top,outside},
    %capbesidewidth=\marginparwidth,
    %capbesideframe=yes,
    %capbesidesep=columnsep,
    %floatrowsep=columnsep,
    %heightadjust=nocaption,
    facing=yes,
    }

%%% CAPTIONS 
\usepackage[]{caption}
\DeclareCaptionLabelSeparator{vbar}{ | } % custom; standard z.B. colon, period, ...
\captionsetup{labelfont=bf,font={sf,footnotesize},format=plain,labelsep=vbar}
  


\bibliographystyle{meinbst}
\usepackage{natbib}

%%%%% Grafiken, Abbildungen
\usepackage{graphicx} % option final to show images also in draft mode
\usepackage{subfig}
\usepackage{wrapfig}
\graphicspath{{Figures/}}

%%%% Tabellen,Listen
\usepackage{tabularx,booktabs,multirow}

%%%% Mathe, Zahlen, chem. Formeln
\usepackage{amsmath,amssymb}
\usepackage{commath}

\usepackage{bm}
\newcommand{\vect}[1]{\bm{#1}} % vector in bold

\usepackage{xfrac} %small fractions

%%%% Verschiedenes
\usepackage[para,multiple,stable,perpage,symbol*]{footmisc}
%%% footnote without marker
\newcommand\blfootnote[1]{%
  \begingroup
  \renewcommand\thefootnote{}\footnote{#1}%
  \addtocounter{footnote}{-1}%
  \endgroup
}

\usepackage{relsize} %size relative to other in sub floats

%%%% Comment-Sections
\usepackage{comment} %% drinnen lassen fuer Lang-Abstract

\usepackage[textsize=scriptsize,bordercolor=none,backgroundcolor=YellowGreen,linecolor=YellowGreen]{todonotes}
%\renewcommand{\todo}[1]{\todo{\sffamily #1}}
\newcommand{\dofig}[1]{\todo[backgroundcolor=DarkSeaGreen,linecolor=none]{\sffamily\textbf{DoFigure:}~#1}\xspace}
\newcommand{\dotxt}[1]{\todo[backgroundcolor=Coral,linecolor=none]{\sffamily\textbf{DoText:}~#1}\xspace}
\newcommand{\doref}[1]{\todo[backgroundcolor=Gold,linecolor=Gold]{\sffamily\textbf{DoRef:}~#1}\xspace}
\newcommand{\doalt}[1]{\textcolor{SkyBlue}{#1}\todo[backgroundcolor=SkyBlue,linecolor=SkyBlue]{\sffamily\textbf{Altrn:}~#1}\xspace}

%%%% Hier kommt's auf die Reihenfolge an
\usepackage{varioref}
\usepackage[pdfpagelabels]{hyperref}
%\usepackage{breakurl} % damit URLs korrekt umgebrochen werden
\usepackage[capitalise]{cleveref}

\hypersetup{%
        pdftitle={Dissertation},    
        pdfauthor={Raul Garcia Diez},
        pdfcreator={pdfLaTeX},
        pdfborder=0 0 0,
        breaklinks=true,
        bookmarksopen=true,
        bookmarksnumbered=true,
        linkcolor=NavyBlue,
        urlcolor=NavyBlue,
        citecolor=NavyBlue,
        colorlinks=false}

%%% Kopf- und Fußzeilen
\newlength{\marginWidth}
\setlength\marginWidth{\marginparwidth+\marginparsep}
\newlength{\fulllinewidth}
\setlength\fulllinewidth{\textwidth+\marginWidth}

\usepackage{truncate} %Um zu lange Kapiteltitel abzuschneiden

\footskip=1.6cm
\makeatletter % = mache @ letter 

%Vordefinition mehrfachverwendeter Teile
\def\oddfootSTANDARD{
   \renewcommand{\@oddfoot}{
       \hbox to\textwidth{\vbox{\hbox to\textwidth{
          \hfill
          \strut
          \hspace{1pt}
       }}}
       \hbox to\marginWidth{\vbox{\hbox to\marginWidth{
          \strut %unsichtbares Zeichen
               \large
               \hspace{5pt}               
               \vrule width 1pt height 1cm
            \hspace{8pt}            
            \textsf{\thepage}
            \hfill
       }}}\hss   
   }
}

\def\evenfootSTANDARD{
   \renewcommand{\@evenfoot}{
      \hspace{-\marginWidth}  
         \hbox to\marginWidth{\vbox{\hbox to\marginWidth{
         \large
         \strut %unsichtbares Zeichen
         \hfill
         \textsf{\thepage}
         \hspace{5pt}
         \vrule width 1pt height 1cm
         \hspace{7pt}
      }}}\hss
   }  
}

%Standardstil für die gesamte Dissertation
\newcommand{\ps@thesis}{%
   \renewcommand{\@oddhead}{%
         \hbox to\textwidth{\vbox{\hbox to\textwidth{%
            \textsf
            \hfill
            \rightmark
            \strut
            \hspace{1pt}
      }}}
         \hbox to\marginWidth{\vbox{\hbox to\marginWidth{%
            \strut %unsichtbares Zeichen
            \hspace{5pt}
            \vrule width 1pt
            \hspace{5pt}
            \textsf
            \thesection
            \hfill
         }}}\hss
   }  
   
   \renewcommand{\@evenhead}{%
      \hspace{-\marginWidth} 
         \hbox to\marginWidth{\vbox{\hbox to\marginWidth{%
            \hfill
            \strut %unsichtbares Zeichen
            \textbf{\textsf{Chapter~\thechapter}}
            \hspace{5pt}
            \vrule width 1pt
            \hspace{7pt}
            \strut
         }}}\hss
         
         \hbox to\textwidth{\vbox{\hbox to\textwidth{%
            \strut %unsichtbares Zeichen
         \truncate{.9\textwidth}{\leftmark}
         \hfill
      }}}\hss
   }
   
   \oddfootSTANDARD   
   \evenfootSTANDARD   
}
%Der PLAIN-Style der Chapter- und Sonderseiten muss redefiniert werden.
\renewcommand{\ps@plain}{%
   \let\@oddhead\@empty
   \let\@evenhead\@empty
   \let\@evenfoot\@empty   
   \oddfootSTANDARD
}
%Spezieller Stil für Inhaltsverzeichnis und Literaturverzeichnis (ohne Nummern wie 0.0 oder B.0)
\newcommand{\ps@thesisINTRO}{%
   \renewcommand{\@oddhead}{%
         \hbox to\textwidth{\vbox{\hbox to\textwidth{%
            \textsf
            \hfill
            \sffamily\rightmark
            \strut
            \hspace{1pt}
         }}}\hss
   } 
   
   \renewcommand{\@evenhead}{%
         \hbox to\textwidth{\vbox{\hbox to\textwidth{%
            \strut %unsichtbares Zeichen
            \truncate{.9\textwidth}{\sffamily\leftmark}
            \hfill
         }}}\hss  
   }
   
   \oddfootSTANDARD   
   \evenfootSTANDARD   
}

%Spezieller Stil für Abbrevatiation and symbols
\newcommand{\ps@thesisSYM}{%
   \let\@oddhead\@empty
   \let\@evenhead\@empty
   \oddfootSTANDARD   
   \evenfootSTANDARD   
}

%Spezieller Stil für Anhänge
\newcommand{\ps@thesisANHANG}{%
   \renewcommand{\@oddhead}{%
         \hbox to\textwidth{\vbox{\hbox to\textwidth{%
            \textsf
            \hfill
            \rightmark
            \strut
            \hspace{1pt}
         }}}
         \hbox to\marginWidth{\vbox{\hbox to\marginWidth{%
            \strut %unsichtbares Zeichen
            \hspace{5pt}
            \vrule width 1pt
            \hspace{5pt}
            \textsf
            \thechapter
            \hfill
         }}}\hss
   }
   
   \renewcommand{\@evenhead}{%
      \hspace{-\marginWidth}  
         \hbox to\marginWidth{\vbox{\hbox to\marginWidth{%
            \hfill
            \strut %unsichtbares Zeichen
            \textbf{\textsf{Appendix~\thechapter}}
            \hspace{5pt}
            \vrule width 1pt
            \hspace{7pt}
            \strut
         }}}\hss
         
         \hbox to\textwidth{\vbox{\hbox to\textwidth{%
            \strut %unsichtbares Zeichen
            \truncate{.9\textwidth}{\leftmark}
            \hfill
         }}}\hss  
   }
   
   \oddfootSTANDARD
   \evenfootSTANDARD
}


\newcommand{\ps@reallyempty}{%
   \let\@oddhead\@empty
   \let\@evenhead\@empty
   \let\@oddfoot\@empty
   \let\@evenfoot\@empty
}

\renewcommand{\chaptermark}[1]{\markboth{\uppercase{\textsf{#1}}}{}}
\renewcommand{\sectionmark}[1]{\markright{\textsf{#1}}}

\makeatother % = mache @ wieder zu nicht-Buchstaben 
\pagestyle{thesis}


%Problem mit den Seitenzahlen und Headern auf leeren Seiten nach Kapiteln:
\let\origdoublepage\cleardoublepage
\newcommand{\clearemptydoublepage}{%
  \clearpage
  {\pagestyle{empty}\origdoublepage}%
}
\let\cleardoublepage\clearemptydoublepage


%%%%%%%%%%%%%%%%%%%%%%
%%%%% eigene Kommandos
\newcommand{\fssmall}[0]{\fontsize{8pt}{8pt}\selectfont}
\newcommand{\fsmedium}[0]{\fontsize{10pt}{10pt}\selectfont}

\newcommand{\figfont}[1]{\fontsize{#1}{#1}\selectfont} %   font size in subfloat figure

