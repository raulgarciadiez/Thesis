%% publication list
\noindent
\pagestyle{empty}
\selectlanguage{ngerman}

\section*{Erklärung}

Es wurden bereits Teile der Dissertation veröffentlicht.
\vspace{2ex}

Liste der Veröffentlichungen, welche in die Dissertation eingeflossen sind:

\begin{enumerate}

    \item R. Garcia-Diez, C. Gollwitzer, M. Krumrey, \emph{J. Appl. Cryst.} \textbf{48}, 20-28 (2015)

        Der Eigenanteil wird mit 50 $\%$ abgeschätzt.
        Ein Teil der Messungen wurde zusammen mit den Ingenieuren Levent Cibik und Stefanie Langer durchgeführt.

    \item R. Garcia-Diez, A. Sikora, C. Gollwitzer, C. Minelli, M. Krumrey, \emph{Eur. Polym. J.} \textbf{81} 641–649 (2016) 

        Der Eigenanteil wird mit 50 $\%$ abgeschätzt.
        Die Probensysteme wurden von Wen-li Wu präpariert ( Gesamtanteil), die finale Interpretation der Analyse wurde mit Wen-li Wu diskutiert ().

    \item R. Garcia-Diez, C. Gollwitzer, M. Krumrey, Z. Varga, \emph{Langmuir} \textbf{32 (3)}, 772-778 (2015)

        Der Eigenanteil wird mit 50 $\%$ abgeschätzt.
        Die Probensysteme wurden von Hiroki Ogawa präpariert ( Gesamtanteil), die finale Interpretation der Analyse wurde mit Hiroshi Okuda diskutiert ().

    \item C. Minelli, R. Garcia-Diez, A. Sikora, C. Gollwitzer, M. Krumrey, A. Shard, \emph{Surf. Interface Anal.} \textbf{46} 663-667 (2014)

        Der Eigenanteil wird mit 25 $\%$ abgeschätzt.
        Die Messungen wurden zusammen mit Christian Gollwitzer durchgeführt (Abschätzung Fremdanteil ).
        Ein Teil der Auswertung (Pixelgröße, Modul-Misalignment) wurde Christian Gollwitzer durchgeführt (Abschätzung Fremdanteil ).

\end{enumerate}


Ich habe an keiner anderen Hochschule oder Fakultät eine Promotionsabsicht eingereicht.

\vspace{3cm}

\noindent Berlin, den 15. November 2016 \hfill Raul Garcia Diez

\cleardoublepage
