%% publication list
\noindent
\pagestyle{empty}
\selectlanguage{ngerman}

\section*{Erklärung}

Es wurden bereits Teile der Dissertation veröffentlicht.
\vspace{2ex}

Liste der Veröffentlichungen, welche in die Dissertation eingeflossen sind:

\begin{enumerate}
    \item \cite{vicent_polymer_2006} 

        Der Eigenanteil wird mit  abgeschätzt.
        Ein Teil der Messungen wurde zusammen mit den Ingenieuren Levent Cibik und Stefanie Langer durchgeführt, was mit einem Gesamtanteil von  abgeschätzt wird.

    \item \cite{vicent_polymer_2006}

        Der Eigenanteil wird mit  abgeschätzt.
        Die Probensysteme wurden von Wen-li Wu präpariert ( Gesamtanteil), die finale Interpretation der Analyse wurde mit Wen-li Wu diskutiert ().

    \item \cite{vicent_polymer_2006}

        Der Eigenanteil wird mit 75 abgeschätzt.
        Die Probensysteme wurden von Hiroki Ogawa präpariert ( Gesamtanteil), die finale Interpretation der Analyse wurde mit Hiroshi Okuda diskutiert ().

    \item \cite{vicent_polymer_2006}

        Der Eigenanteil wird mit  abgeschätzt.
        Die Messungen wurden zusammen mit Christian Gollwitzer durchgeführt (Abschätzung Fremdanteil ).
        Ein Teil der Auswertung (Pixelgröße, Modul-Misalignment) wurde Christian Gollwitzer durchgeführt (Abschätzung Fremdanteil ).

    \item \cite{vicent_polymer_2006}

        Der Eigenanteil wird mit  abgeschätzt.
        Ein Teil der Messungen wurden von Michael Krumrey durchgeführt ().
        Die Ellipsometriemessungen wurden von Alex G. Shard durchgeführt und ausgewertet ().

    \item \cite{vicent_polymer_2006}

        Der Eigenanteil wird mit  abgeschätzt.
        Dies beinhaltet die Durchführung, Auswertung und Aufbereitung der GISAXS-Experimente sowie Unterstützung bei der Interpretation und Verfeinerung der Modellierung mit dem Maxwell-Solver.

    \item \cite{vicent_polymer_2006}

        Der Eigenanteil wird mit  abgeschätzt.
        Dies beinhaltet die Durchführung, Auswertung und Aufbereitung der GISAXS-Experimente sowie Unterstützung bei der Interpretation und Verfeinerung der Modellierung mit dem Maxwell-Solver.

\end{enumerate}


Ich habe an keiner anderen Hochschule oder Fakultät eine Promotionsabsicht eingereicht.

\vspace{3cm}

\noindent Berlin, den 15. November 2016 \hfill Raul Garcia Diez

\cleardoublepage
