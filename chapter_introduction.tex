\chapter{Introduction: Nanoparticles in medicine and biology}
The morphology of nanoparticles determines the properties necessary for their utilization in real-world applications. For instance, in drug delivery devices the phenomena involved in biocompatibility reactions (e.g. protein adsorption) depend on the amount of available surface and the nanoparticles' properties \citet{vittaz_effect_1996}. Particularly, polymer lattices and biodegradable nanoparticles have been of growing importance of late as drug carriers \citet{kattan_phase_1992} and thus extensively characterized \citet{soppimath_biodegradable_2001}. The size determination and the characterization of the radial structure of the particles are therefore fundamental tasks. 
\label{chap:introduction}

\section{Polymeric colloids}
\subsection{Functionalization for protein binding}

\subsection{Polymerization consequences}
initiator, co-monomer, surfactants

\section{Liposomal nanocarriers}
formation from amphiphilic lipids
\subsection{Phospholipid bilayer}
typical lipid HSPC, DPPC, cholesterol, PEG

\subsection{polydispersity control}
extrusion, paper with zoltan about scattering in SSLs

\subsection{Drug carrier and SSLs}
stealth function, bilayer stability, filling with pH gradient

\section{Physicochemical characterization}
\subsection{Dimensional metrology and traceability}

\subsection{Characterization tools}

The most widely used technique for size determination in the field of nanomedicine is dynamic light scattering (DLS), which measures the hydrodynamic diameter of the nanoparticles (NPs) \citep{murphy_static_1997, hallett_vesicle_1991, egelhaaf_determination_1996, takahashi_precise_2008, jans_dynamic_2009, hoo_comparison_2008}. DLS is well-established and has indisputable advantages in the size characterization of the NPs, e.g. easy-to-use instrumentations, fast and low-cost operation, but it is not capable of a traceable size determination as there is no general relationship between the hydrodynamic diameter and the physical size of the NPs \citep{meli_traceable_2012}. 

Transmission electron microscopy (TEM) is also frequently used for sizing of NPs and proved to be an appropriate technique for solid nanoparticles, whilst its employment in soft matter NPs (e.g. liposomes, micelles and polymeric nanoparticles) is questionable due to the possible distortion of the particles during the drying process.  Although cryo-TEM could overcome this limitation \citep{li_doxorubicin_1998}, the statistical accuracy of this non-ensemble method is usually not sufficient.

\subsubsection{Single-particle method}
AFM, TEM, SEM, TSEM

\subsubsection{Ensemble methods}
DLS, DCS, SAXS


