\chapter{Introduction: Nanoparticles in medicine and biology}
The morphology of nanoparticles determines the properties necessary for their utilization in real-world applications. For instance, in drug delivery devices the phenomena involved in biocompatibility reactions (e.g. protein adsorption) depend on the amount of available surface and the nanoparticles' properties \citet{vittaz_effect_1996}. Particularly, polymer lattices and biodegradable nanoparticles have been of growing importance of late as drug carriers \citet{kattan_phase_1992} and thus extensively characterized \citet{soppimath_biodegradable_2001}. The size determination and the characterization of the radial structure of the particles are therefore fundamental tasks. 
\label{chap:introduction}

\section{Polymeric colloids}
\subsection{Functionalization for protein binding}

\subsection{Polymerization consequences}
initiator, co-monomer, surfactants

\section{Liposomal nanocarriers}
formation from amphiphilic lipids
\subsection{Phospholipid bilayer}
typical lipid HSPC, DPPC, cholesterol, PEG

\subsection{polydispersity control}
extrusion, paper with zoltan about scattering in SSLs

\subsection{Drug carrier and SSLs}
stealth function, bilayer stability, filling with pH gradient

\section{Physicochemical characterization}
\subsection{Dimensional metrology and traceability}

\subsection{Characterization tools}
\subsubsection{Single-particle method}
AFM, TEM, SEM, TSEM

\subsubsection{Ensemble methods}
DLS, DCS, SAXS


