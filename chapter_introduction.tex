\chapter{Introduction}
\label{chap:introduction}

%There have been few pivotal moments in history when science fiction became science fact and fundamentally reshaped the use of technology in our society\textcolor{blue}{/increased the impact of technology on society/opened a new technological frontier/brought technology to the next level/altered forever the way technology is perceived}. 

In 1966, Richard Fleischer directed \emph{Fantastic Voyage}, a film about the voyage of a miniaturized submarine used to cruise along human blood vessels and repair the damaged caused to the scientist's brain by a blood clot. The idea of treating from the inside damaged cells or organs fuelled the imagination of the next generation scientists and shaped the incipient field of nanomedicine. Less than 30 years later, science fiction became science fact and Doxil was approved by the US Food and Drug Administration in 1995 as the first nano-drug comercially available. Although 20 years after this milestone nano-submarines are still a long way off, nanomedicine is a well-established research field and dozens of products are under clinical trials or have been approved by the relevant health agencies.

The precursor to the nanomedicine breakthrough can be found in the tremendous progress in nanoparticles research observed in the 60s and 70s of the last century. Nanoparticles (NPs) are objects with \emph{one or more external dimensions in the size range from 1 nm to 100 nm} (European Commission Recommendation for nanomaterial (2011/696/EU)) and have a preeminent position in the continuously growing world of nanotechnology, employed as paints or cosmetic products \citep{guterres_polymeric_2007}. Besides, the applcation of NPs in the emerging field of nanomedicine opens up exciting prospects, especially considering their possibilities as platforms for drug-delivery \citep{wang_nanoparticle_2012} or encapsulating imaging agents \citep{tao_shape-specific_2011}.

The development of NPs is currently focused towards tailoring nano-drug carriers with flexible surface functionalisations and controlled morphologies \citep{euliss_imparting_2006,yang_shape-memory_2005}. The morphology of NPs is typically specified by parameters like size, shape, density or chemical composition of the particle, which are fundamental and defining aspects of the particle functions and determine their applicability in real-world medical applications. In this regard, the size of NPs is one of the most crucial phsicochemical properties of nanodrugs, because it delimits whether they can intrude into the biological cells or the targeted tumor sites. An accurate and reliable description of the morphological traits of the NPs is therefore of vital importance for their favorable translation into successful nanomaterials.

The term \emph{nanometrology} refers to the science of accurate and correct measurement of relevant properties at the nanometre range. A central concept in metrology is \emph{traceability}, which refers to the ability of relating the measurand i.e. measured value to a base unit definition of the Internation System of Units (SI system) by an unbroken chain of comparisons with known uncertainties. This enables an objective comparison of the results obtained by different methods based on a consistent uncertainty budget associated to the measurand. The basic research in the field of metrology in Germany is addressed by its national metrology institute, the Physikalisch-Technische Bundesanstalt (PTB). Founded in 1887, the PTB is devoted among other metrological activities to the determination of fundamental and natural constants or the technology transfer for the industry.

In the nanoscale level, PTB is involved in the development of the dimensional nanometrology field, which studies the measurement of the physical size or distances of a given nanomaterial and traces it back to the \emph{nm} unit. There are several available techniques which are suitable for the sizing of NPs, though not all provide a traceable measurement. A prime example is dynamic light scattering (DLS), the most widely used tool in nanomedicine \citep{murphy_static_1997, hallett_vesicle_1991, egelhaaf_determination_1996, takahashi_precise_2008, jans_dynamic_2009, hoo_comparison_2008}. DLS is well-established and has indisputable advantages in the size characterization of the NPs, e.g. easy-to-use instrumentations, fast and low-cost operation, but it is not capable of a traceable size determination as there is no general relationship between the measured hydrodynamic diameter and the physical size of the NPs \citep{meli_traceable_2012}.

Other ensemble techniques extensively used are differential centrifugal sedimentation (DCS) \citep{fielding_correcting_2012} and Nanoparticle Tracking Analysis (NTA), both capable as well of measuring the NPs in suspension, i.e. in their native medium. While DCS is based on the sedimentation of NPs through a density gradient, NTA is a single-particle counting method that relates the Brownian movement of the particles with the measured laser light scattering. The particle size distribution obtained with DCS is traced down to a calibration standard of known size and density, resulting in a non-traceable measurement. Similarly to DLS, the NTA measured uses the hydrodynamic properties of the NPs for its calculation (REF-Towards traceable size determination of extracellular vesicles).

Microscopic tools are also frequently used for structural investigations and proved to be an appropriate technique for solid NPs \citep{joensson_morphology_1991,silverstein_microstructure_1989} due to its traceability achieved by the coupling of a laser interferometer with the measurement table. Nevertheless, techniques such as transmission scanning electron microscopy (TSEM) or atomic force microscopy (AFM) are not ensemble averaged and the statistical accuracy of non-ensemble methods is usually not sufficient. Another drawback is the removal of the suspending medium and the possible distortion of the particle's morphology during the drying process, which can be partially overcome by cryo-TEM \citep{li_doxorubicin_1998}.

\section{advantadges to other characterization tools}


Transmission electron microscopy (TEM) is also frequently used for sizing of NPs and proved to be an appropriate technique for solid nanoparticles, whilst its employment in soft matter NPs (e.g. liposomes, micelles and polymeric nanoparticles) is questionable due to the possible distortion of the particles during the drying process.  Although cryo-TEM could overcome this limitation \citep{li_doxorubicin_1998}, the statistical accuracy of this non-ensemble method is usually not sufficient.

The characterisation of polymeric nanoparticles remains a challenge due to their typically complicate internal structure\citep{beyer_saxs_1990} and requires more than a single characterisation technique to detect these heterogenous compositions. For instance, electron microscopy is an effective tool for direct observation of the shape and size distribution of nanoparticles, although it cannot conclusively elucidate their internal morphology.




Several reasons for these discrepancies are possible, as the different methods use different physical properties of the particles for the size determination like interactions with photons or electrons or the mean velocity of the particles in different liquids.

In addition, a form factor model is fitted to the scattering curves to obtain decisive information about the internal morphology of the particle, which is not directly available by other techniques such as transmission scanning electron microscopy (TSEM), differential centrifugal sedimentation (DCS)\citep{fielding_correcting_2012} or atomic force microscopy (AFM).

\section{SAXS is a good tool: small internal electron density differences}

Small-angle X-ray scattering is a powerful technique that can elucidate the structural features of particles with sizes ranging from about 2 nm up to 400 nm. By investigating the photons elastically scattered by the electron density distribution of the particle $\rho_e ( \vect{r} )$, the resulting patterns can be analysed employing equation \ref{eq:rayleigh_cross_section} to obtain information about the particle size, shape and composition.

The use of an ensemble-average and non-destructive technique such as small-angle X-ray scattering (SAXS) arises as an appropiate alternative\citep{leonard_jr_size_1952,motzkus_untersuchung_1959}. SAXS can discern differences in the radial structure of polymeric colloids and offers advantages to other methods which require prior treatment of the sample and are not averaging\citep{silverstein_microstructure_1989,joensson_morphology_1991}. Despite being a highly informative method for the accurate characterisation of polymeric particles, the difficulties in the interpretation of the scattering curves demands complementary experimental information\citep{mykhaylyk_structural_2012}.

SAXS is capable of traceable size determination for sufficiently monodisperse nanoparticles \citep{meli_traceable_2012} and therefore the continuous contrast variation technique with SAXS is a suitable method to assess the size of a complex liposomal drug, such as the PEGylated liposomal formulation of doxorubicin. The need of an iso-osmolal suspending medium to mimic the phyisiological conditions of plasma requires the use of Optiprep \textregistered as contrast agent, an aqueous solution of iodixanol, which has an osmolality of 290 to 310 mOsm kg$^{-1}$. By employing Optiprep \textregistered, the suspending medium osmolality can be kept constant along the density gradient capillary.

Using X-ray synchrotron radiation allows advanced scattering techniques such as chemical element sensitivity by anomalous scattering at photon energies close to an absorption edge.




Besides, the ability of this technique to determine the density of polymeric colloids in suspension is also discussed. Normally, the density of the suspended particles can not be compared to the bulk density of the dry material. Such a complex question has been adressed by different methods, though with evident limitations. For example, the density of polymeric beads has been measured previously with field-flow fractionation (FFF) with high-accuracy but at the expense of \emph{a priori} assumptions about the morphology of the particle\citep{giddings_density_1981,yang_colloid_1983,caldwell_measurement_1986}. Assuming the Stokes' diameter as the actual size of the colloid, recent advances in analytical ultracentrifugation allow the complementary characterisation of the size, density and molecular weight of gold nanoparticles\citep{carney_determination_2011}.

The density of the 3 polymeric colloids was also analysed by DCS and the results compared and discussed with those obtained by SAXS. DCS uses the sedimentation of particles through a density gradient to measure high resolution particle size distributions\citep{minelli_characterization_2014}. Its accuracy typically depends on the knowledge of the density of the particles. When the size of the particle is known, DCS can alternatively be used to measure average particle's density.

In this study, the size and density of low-density particles is independently determined by performing DCS measurements with two different discs using the sedimentation and flotation respectively of the particles through a density gradient and solving the relative Stokes' equations. A similar approach to DCS which combines the results of two independent measurements has been investigated previously. For example, \cite{neumann_new_2013} used two sucrose gradients resulting in different viscosities and densities, where the altered settling velocity combined with linear regression analysis was used for the calculation of the size and density of silica nanoparticles and viruses. \cite{bell_emerging_2012} adopted a two gradient method based on the variation of the sucrose concentration to determine the density of the St\"ober silica and the calibration standards used in DCS.

\section{contrast variation is even better, but takes too long and extra uncertainties arise}

In the contrast variation method, the electron density of the particle or the surrounding medium is systematically altered in order to obtain independent scattering curves with different contrasts $\Delta \eta (r)$. This technique is useful to characterize the different components of a heterogenous particle, due to the complementary data that can be collected at each contrast. The work presented in this thesis is focused in the \emph{solvent contrast variation} method, where only the electron density of the suspending medium is varied.

By means of the solvent contrast variation approach, the electron density of a single phase of the investigated particle can be matched (i.e. \emph{match point}), resulting in a increased scattering amplitude of the other components of the object, as depicted in figures \ref{fig:ContrastScheme} and \ref{fig:MatchPointScheme}. This effect enables a much more detailed study of the different contributions of the particle's components to the scattering intensity, which can be isolated by choosing the solvent electron density appropriately. In the following paragraphs, the theoretical framework needed to interpret a SAXS contrast variation experiment will be presented, focusing mainly on the effects produced by the variation of the solvent electron density $\rho_{\text{solv}}$.


The contrast variation method in SAXS varies systematically the electron density of the suspending medium by adding a suitable contrast agent, e.g. sucrose, in order to resolve the different contributions of the particle components to the scattering. By measuring SAXS patterns as a function of the adjusted contrast, a more detailed insight into the particle morphology can be obtained in comparison to single-contrast experiments\citep{bolze_situ_2004}. For instance, the internal structure can be modelled in terms of the radial electron density\citep{dingenouts_radial_1994,dingenouts_analysis_1999,ballauff_analysis_2011,ballauff_small-angle_1996} and the individual contribution of each polymer can be distinguished\citep{beyer_saxs_1990,grunder_analysis_1991,grunder_small-angle_1993,ottewill_characterization_1995,bolze_small-angle_1997,dingenouts_structure_1994} as well as its density\citep{mykhaylyk_application_2007}.


The contrast variation method in Small Angle X-ray Scattering (SAXS) experiments consists in systematically varying the electron density of the dispersing media to study the different contributions to the scattering intensity in greater detail as compared to measurements at a single contrast. It emerges as an ideally suited technique to elucidate the structure of particles with a complicated inner composition and has been repeatedly employed to investigate the radial structure of nanoparticles in suspension, e.g. latex particles suspended in an aqueous medium \citep{dingenouts_analysis_1999,ballauff_analysis_2011}. 

In SAXS, the solvent contrast variation technique is achieved by adding a suitable contrast agent to the suspending medium (e.g. sucrose) and recording the scattering data as a function of the adjusted solvent electron density \( \rho_{\text{solv}} \) \citep{ballauff_saxs_2001-1,bolze_application_2003}. In order to resolve small changes of the radial structure, the average electron density of the colloidal particles must be close to the dispersant's, i.e., the \emph{match point} should be approached, where the average contrast of the particle vanishes. In the case of polymeric lattices with electron densities ranging from 335 to \(390 \mbox{ nm}^{-3}\), an aqueous sucrose solution is very well suited as the suspension medium, due to the easy realization of concentrated solutions with electron densities of up to \(400 \mbox{ nm}^{-3}\). Previous studies on globular solutes \citep{kawaguchi_isoscattering_1992} and the influence of the sucrose on the size distribution of vesicles \citep{kiselev_sucrose_2001-1} show the feasibility of this technique, while further studies have investigated the effect of the penetration of the solvent into the particles \citep{kawaguchi_isoscattering_1993}.


\section{density gradient technique: tunable and large datasets}

The formation of a solvent density gradient within a capillary emerges as an intelligent strategy to measure SAXS patterns at a continuous range of contrasts and, as a result, collect in a relatively short timespan an extensive data set of complementary scattering curves\citep{garcia-diez_nanoparticle_2015}. The continuous contrast variation technique in SAXS is ideally suited for current synchrotron radiation sources, where high brilliance and collimation permit the measurement of the scattering curves within the diffusion time of the contrast agent.

The preparation of a number of different sucrose solutions has been a major inconvenience in solvent contrast variation experiments, due to the tedious, time-consuming process, possible inaccuracy in the sucrose concentration and the discrete range of available solvent electron densities. In this chapter, a novel approach using a density gradient column is introduced, which allows the tuning of the solvent contrast within the provided density range, resulting in a virtually continuous solvent contrast variation. By filling the bottom part of the capillary with a particle dispersion in a concentrated sucrose solution and the top part with an aqueous solution of the same particle concentration, a solvent density gradient is initiated with a constant concentration of nanoparticles along the capillary. Density gradient columns are extensively used in fields like marine biology \citep{coombs_density-gradient_1981} or biochemistry together with centrifugation \citep{hinton_density_1978}, to create a continuously graded aqueous sucrose solution by diffusion of the sucrose molecules. By measuring the density gradient column at different points in time during the diffusion process of the sucrose, it is possible to choose \emph{in situ} the most appropriate solvent densities to perform measurements close to the contrast match point. Combining this approach with SAXS, a very extensive dataset with a virtually continuous variation in the suspending medium density can be acquired in a short interval of time.

This work demonstrates the simultaneous size and density determination using continuous contrast variation technique in SAXS with 3 polymeric particles of different sizes and polymeric species. By means of an aqueous sucrose density gradient, the measurements were achieved along a large range of suspending medium densities, from water density to that of poly(methyl methacrylate)'s, highlighting the relevance of the technique across a wide spectrum of polymers. This chapter discusses the applicability of this method for the traceable size determination of these colloids, where a high-resolution size distribution of the particles is presented. Focusing on a low-density colloid, different evaluation approaches to SAXS contrast variation experiments are discussed and the advantages and drawbacks of a model-free formulation like the isoscattering point position are discussed, together with the accuracy of the scattering shape factor.


\section{polymeric particles: why important?}

The most recent efforts in nanomedicine aim for a high control of the characteristics of the nanocarrier surface, as the surface's properties are a defining element of its efficiency as drug carrier. Besides, nanoparticles interact with proteins when introduced into biological media, leading to the formation of the so-called \emph{protein corona} surrounding the nanocarrier \citep{cedervall_understanding_2007,monopoli_physicalchemical_2011,casals_time_2010}. The identity of the biomolecule coating depends on the particle size, surface functionalization and charge \citep{lundqvist_nanoparticle_2008,tenzer_rapid_2013,gessner_functional_2003} and its detailed description is challenging. Yet, the ability to quantitatively characterise this interface is important in understanding particle behaviour in these complex environments and improving their surface engineering for enhanced functionality.

\section{polymeric: extensively used in SAXS and contrast variation, also SANS because deuteration}

In Small Angle Neutron Scattering (SANS) the contrast variation technique is widely used by mixing water and deuterium oxide, but the use of deuterated chemicals and the incoherent contribution to the background as well as the limited access to neutrons restrict the application of this technique.

\section{polymeric: good application because of phase separation in polymerization}

initiator, co-monomer, surfactants

\section{bio materials: important why?}

In the continuously growing world of nanotechnology, nanoscience provides understanding for biological structures at the nanometer length scale, such as lipoprotein biology, while the application of nanoparticles in medicine opens exciting new possibilities in this field \citep{nie_nanotechnology_2007, sahoo_nanotech_2003, wickline_nanotechnology_2003, zhou_nano-enabled_2014, rosen_rise_2005}. For example, polymeric colloids and other biodegradable nanocarriers are finding many medical applications\citep{vicent_polymer_2006} and are starting to undergo clinical trials\citep{patel_polymeric_2012,beija_colloidal_2012,cabral_progress_2014}. 

From a nanoscience point of view, human blood can be seen as a suspension of particles with different physiological roles, where important components are in the nanorange. Serum lipoproteins are the colloidal particles involved in the transport and metabolism of insoluble lipids and are among the most studied biological particles. The interest in their activity is understandable due to their direct relationship with very extended diseases in the Western world population, such as obesity or atherogenesis, e.g. obturation of the arterial walls. For example, the dysregulation of cholesterol in plasma, primarly carried within lipoproteins, is responsible of atherosclerosis \citep{munro_pathogenesis_1988}. Besides, they are a convenient model for lipid-protein interactions \citep{assmann_lipid-protein_1974} due to their lipid core and the hydrated proteins isometrically situated on its surface.

Lipoproteins are isolated from blood plasma by ultracentrifugation \citep{havel_distribution_1955} and are normally classified by their density range, showing different chemical composition, size and pathological condition for each class \citep{german_lipoproteins:_2006}. Indeed, the size of lipoproteins is critically connected with disease risk \citep{gardner_cd_association_1996} and Low-density Lipoproteins (LDL) are suggested to be more or less atherogenic depending on their size \citep{dreon_low-density_1994}. The effect of diabetes on the lipoprotein size is also of great interest, especially the sex-dependency of High-density Lipoproteins (HDL) size \citep{colhoun_lipoprotein_2002}.

Therefore, precise sizing techniques are a crucial tool to understand the physiological processes of lipoproteins \citep{german_lipoproteins:_2006}. The naturally narrow size distributions of LDL and HDL suggest small-angle scattering as a well-suited method and their heterogenous morphology advises the use of a contrast variation approach. For instance, the first characterization attempts date back to the late 1970s with neutron scattering \citep{stuhrmann_neutron_1975}, using salt \citep{tardieu_structure_1976} and sucrose \citep{muller_structure_1978} as SAXS contrast agents or modifying the sample temperature \citep{laggner_molecular_1977,luzzati_structure_1979}. 

\section{liposomal nanocarriers: what are they and why important? How are they characterized?}

In this sense, lipid vesicles, or liposomes, have an increasing importance in the emerging field of nanomedicine, due to their capacity to encapsulate hydrophilic compounds within the closed phospholipid bilayer membrane. The first approved nanodrug, Caelyx\textregistered, consists of a PEGylated liposomal formulation of doxorubicin\citep{barenholz_doxil_2012}. In fact, liposomal nanocarriers are nowadays a widespread instrument for drug delivery\citep{perez-herrero_advanced_2015}.

formation from amphiphilic lipids

The first approved nano-drug was Doxil\textregistered (Caelyx\textregistered in Europe), a PEGylated liposomal formulation of doxorubicin, which was followed by a few other products\citep{yeh_clinical_2011,barenholz_doxil_2012}. Nowadays there are approximately 250 nanomedicine products that are either approved by the relevant health agencies or are under clinical trials \citep{etheridge_big_2013}. On the other hand, there is a translational gap between the experimental work devoted to the development of new nano-drug candidates and the clinical realization of their use, which is also reflected in the high number of studies dealing with nanomedicine and the number of approved products on the market \citep{khorasani_closing_2014, venditto_cancer_2013}. As highlighted in a recent review by \citep{khorasani_closing_2014}, one of the main reasons for this translational gap is that the current characterization techniques possess limitations and there is a need for standardization in this field.

\section{liposomes: SAXS characterization, literature}

Despite SAXS being a usual method of choice for the accurate characterization of nanomaterials, the interpretation of the scattering curves, i.e. the model fitting, is frequently intricate for complex samples. Liposomal drugs or loaded polymeric nanoparticles belong to this class, as both the carrier and the incorporated biotarget contribute to the scattering intensity. These heterogenous samples require either \emph{a priori} knowledge about their morphology or the measurement of complementary scattering curves obtained under different experimental conditions, like in solvent contrast variation in SAXS


The rigidity of the nanocarriers is a relevant property directly related with its drug delivery efficacy, the particle stability or the release rate of the encapsulated drug. In fact, some of these characteristics might change upon injection due to the osmotic pressure applied to the nanocarriers in the process. In the case of lipid vesicles, i.e. liposomes, the permeability of water through the phopholipid bilayer is a defining aspect of their physicochemical behavior. Although many aspects about the membrane permeability have been studied \citep{nagle_theory_2008, mathai_structural_2008, olbrich_water_2000}, the evaluation of the liposomes rigidity and its osmotic activity is still challenging.

The osmotic behavior of liposomes depends, basically, on their size and chemical composition. For example, the incorporation of cholesterol can vary the fluidity of the lipid bilayer. Larger liposomes tend to be osmotically active \citep{de_gier_osmotic_1993} and behave according to the Laplace law: the osmotic pressure needed to deform them decreases for increasing sizes. In the case of liposomal nanocarriers, the intraliposomal osmolality should be equal to the buffer outside of the liposomes to enhance the particle stability. 

Therefore, it is an important question whether the incorporation of a drug into the intraliposomal volume might modify its osmotic activity. For example, it is expected that the small size of Caelyx\textregistered and the doxorubicin-sulfate aggregate in the intraliposomal volume increase the resistance against the buffer osmotic pressure in comparison to an empty liposomal particle. Unfortunately, no osmotic pressure effects were observed in the size or density of the liposomal drug Caelyx \textregistered in the previous section \ref{sec:caelyx_size} due to the constant osmolality of the suspending medium along the whole density gradient.

\section{liposomes: effects of PEG, cholesterol content, long discussion}

typical lipid HSPC, DPPC, cholesterol, PEG

extrusion, paper with zoltan about scattering in SSLs

Typically, unilamellar liposomes present a very narrow size distribution and spherical shape, whose diameter ranges from 50 nm to some hundreds of nanometers. The covalent attachment of biocompatible polymers can improve the liposome stability. For example, polyetyhlene glycol (PEG) shows very low toxicity \citep{yamaoka_distribution_1994} and is a widely used stabilizer \citep{sou_polyethylene_2000}. PEGylated liposomal formulations, also called sterically stabilized liposomes (SSL) or \emph{stealth} liposomes, show longer blood circulation times \emph{in vivo} \citep{barenholz_liposome_2001} and exhibit a slow drug release rate. PEG-modified liposomes have become of importance lately due to their increased drug pharmakinetics, decreased plasma clearance and improved patient convenience \citep{gabizon_polyethylene_1997,harris_effect_2003}. Therefore, the self-assembly of lipid structures in the presence of PEG moieties has been studied for different lipids \citep{lee_coarse-grained_2011}.

The incorporation of biocompatible polymers increases the phospholipid bilayer strength and enhances the vesicle rigidity, which relates to the increase of the bending modulus \citep{liang_effect_2005, sou_polyethylene_2000}. The higher membrane stiffness of SSLs has been extensively characterized with methods such as Atomic Force Microscopy (AFM) \citep{spyratou_atomic_2009} though other techniques such as light scattering have found a higher osmotic activity in SSLs in comparison to their non-PEGylated counterparts when incubated in serum \citep{wolfram_shrinkage_2014}. Further investigations about the relationship between PEGylation and the liposomal osmotic behavior in suspension are essential. In the following work, the different response of SSLs and plain liposomes to osmotic pressure is studied with SAXS. 

\section{liposomal nanocarrier: SAXS characterization, literature, micellesm other bio-drugs}

stealth function, bilayer stability, filling with pH gradient

\section{structure of the thesis}

In this thesis, an overview of nanoparticles and current nanometrology is given in chapter 1. In chapter 2 Small-Angle X-ray Scattering and its application for the determination of the size of nanoparticles is explained. The tools and instruments used for the measurements carried out in the course of this thesis are described in chapter 3. Chapter 4 details how the measurements and analyses were performed
in general. The results for the different samples are presented in chapter 5. Finally, chapter 5.4 comprises a final discussion and the conclusion of this thesis.

Extensive parts of the work presented in Chapters 3 to 6 have been published in peer-reviewed journals.[REF] Annotations in each (sub-)chapter indicate which publications correspond to the presented results.

Large parts of every sub-chapter are published in peer-reviewed journals.

