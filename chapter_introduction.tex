\chapter{Introduction: Nanoparticles in medicine and biology}
The morphology of nanoparticles determines the properties necessary for their utilization in real-world applications. For instance, in drug delivery devices the phenomena involved in biocompatibility reactions (e.g. protein adsorption) depend on the amount of available surface and the nanoparticles' properties \citet{vittaz_effect_1996}. Particularly, polymer lattices and biodegradable nanoparticles have been of growing importance of late as drug carriers \citet{kattan_phase_1992} and thus extensively characterized \citet{soppimath_biodegradable_2001}. The size determination and the characterization of the radial structure of the particles are therefore fundamental tasks. 

\section{Size and density determination: Intro}
In the continuously growing world of nanotechnology, nanonoparticles have a preeminent position, employed as pharmaceutical or cosmetic products\citep{guterres_polymeric_2007} and especially in the emerging field of nanomedicine. Indeed, nanoparticles open exciting new possibilities in this field as platforms for drug-delivery\citep{wang_nanoparticle_2012} or encapsulating imaging agents\citep{tao_shape-specific_2011}. Nowadays, polymeric colloids and biodegradable nanocarriers are finding many research and medical applications\citep{vicent_polymer_2006} and are starting to undergo clinical trials\citep{patel_polymeric_2012,beija_colloidal_2012,cabral_progress_2014}.

The use of an ensemble-average and non-destructive technique such as small-angle X-ray scattering (SAXS) arises as an appropiate alternative\citep{leonard_jr_size_1952,motzkus_untersuchung_1959}. SAXS can discern differences in the radial structure of polymeric colloids and offers advantages to other methods which require prior treatment of the sample and are not averaging\citep{silverstein_microstructure_1989,joensson_morphology_1991}. Despite being a highly informative method for the accurate characterisation of polymeric particles, the difficulties in the interpretation of the scattering curves demands complementary experimental information\citep{mykhaylyk_structural_2012}.

The contrast variation method in SAXS varies systematically the electron density of the suspending medium by adding a suitable contrast agent, e.g. sucrose, in order to resolve the different contributions of the particle components to the scattering. By measuring SAXS patterns as a function of the adjusted contrast, a more detailed insight into the particle morphology can be obtained in comparison to single-contrast experiments\citep{bolze_situ_2004}. For instance, the internal structure can be modelled in terms of the radial electron density\citep{dingenouts_radial_1994,dingenouts_analysis_1999,ballauff_analysis_2011,ballauff_small-angle_1996} and the individual contribution of each polymer can be distinguished\citep{beyer_saxs_1990,grunder_analysis_1991,grunder_small-angle_1993,ottewill_characterization_1995,bolze_small-angle_1997,dingenouts_structure_1994} as well as its density\citep{mykhaylyk_application_2007}.

The formation of a solvent density gradient within a capillary emerges as an intelligent strategy to measure SAXS patterns at a continuous range of contrasts and, as a result, collect in a relatively short timespan an extensive data set of complementary scattering curves\citep{garcia-diez_nanoparticle_2015}. The continuous contrast variation technique in SAXS is ideally suited for current synchrotron radiation sources, where high brilliance and collimation permit the measurement of the scattering curves within the diffusion time of the contrast agent.

\label{chap:introduction}

\section{Polymeric colloids}
\subsection{Functionalization for protein binding}



\subsection{Polymerization consequences}
initiator, co-monomer, surfactants

\section{Liposomal nanocarriers}
formation from amphiphilic lipids
\subsection{Phospholipid bilayer}
typical lipid HSPC, DPPC, cholesterol, PEG

\subsection{polydispersity control}
extrusion, paper with zoltan about scattering in SSLs

\subsection{Drug carrier and SSLs}
stealth function, bilayer stability, filling with pH gradient

\section{Physicochemical characterization}
\subsection{Dimensional metrology and traceability}

\subsection{Characterization tools}

\subsubsection{DLS and TEM: Characterization for soft-matter, Caelyx}
The most widely used technique for size determination in the field of nanomedicine is dynamic light scattering (DLS), which measures the hydrodynamic diameter of the nanoparticles (NPs) \citep{murphy_static_1997, hallett_vesicle_1991, egelhaaf_determination_1996, takahashi_precise_2008, jans_dynamic_2009, hoo_comparison_2008}. DLS is well-established and has indisputable advantages in the size characterization of the NPs, e.g. easy-to-use instrumentations, fast and low-cost operation, but it is not capable of a traceable size determination as there is no general relationship between the hydrodynamic diameter and the physical size of the NPs \citep{meli_traceable_2012}. 

Transmission electron microscopy (TEM) is also frequently used for sizing of NPs and proved to be an appropriate technique for solid nanoparticles, whilst its employment in soft matter NPs (e.g. liposomes, micelles and polymeric nanoparticles) is questionable due to the possible distortion of the particles during the drying process.  Although cryo-TEM could overcome this limitation \citep{li_doxorubicin_1998}, the statistical accuracy of this non-ensemble method is usually not sufficient.


\subsubsection{Electron Microscopy: Polymeric NPs, chapter 6}
The characterisation of polymeric nanoparticles remains a challenge due to their typically complicate internal structure\citep{beyer_saxs_1990} and requires more than a single characterisation technique to detect these heterogenous compositions. For instance, electron microscopy is an effective tool for direct observation of the shape and size distribution of nanoparticles, although it cannot conclusively elucidate their internal morphology.


\subsubsection{Single-particle method}
AFM, TEM, SEM, TSEM

\subsubsection{Ensemble methods}
DLS, DCS, SAXS


