\chapter{Summary and outlook} 
\label{chap:conclusions}

\section{chap4}

From these results, it is evident that the radius of gyration interpretation produces the most deviant values. This might be founded in the complicated function fitted to the data and the reduced availability of $q$-range employed to obtain \( R_g \). The resulting polydispersity degree of the measured particles from the model fit is in agreement with the upper limit obtained with the radii of gyration. Nevertheless the polydispersity is the parameter determined with the largest uncertainty in the fitting process and therefore this result must be considered with care.

It can be concluded that the different approaches show consistent and complementary results about the size distribution of nanoparticles with radial inner structure, especially for the external radius of the particle and its average electron density. A precise value for the polydispersity degree could not be obtained as explained previously, although a credible upper limit to the polydispersity degree of $24\,\%$ could be given.

This work demonstrates that it is possible to perform continuous contrast variation for light nanoparticles by means of a density gradient and to collect a large quantity of SAXS curves, which can be analyzed with complementary approaches to reveal a consistent insight into the size distribution and the inner structure of the suspended nanoparticles. 

By a model-free analysis of the experimental data based on the isoscattering point theory, an average particle diameter of 101 nm was obtained. The analysis of the Guinier region of the scattering curves shows that the radial inner structure of the particles consists of a thin, more dense layer coating the polystyrene core. Complementing these results, a core-shell model fit showed that the core component of the particle had exactly the same electron density expected for polystyrene and the shell was composed of a compound with a density below that of PMMA. This core-shell structure was expected for chemical reasons due to the different hydrophobicity of PS compared to MMA and MAA.

Besides, the precision in the determination of the electron density by different approaches proves this technique as a useful tool and an alternative to other techniques like isopycnic centrifugation \citep{vauthier_measurement_1999,arnold_sorting_2006,sun_separation_2009}, widely used with biomacromolecules. A more comprehensive discussion about the suitability of the proposed method for density measurements is enclosed in chapter \ref{chap:simultaneous_size_density}, where the technique is applied to a several polymeric nanoparticles.

\section{chap5}

This work demonstrates how continuous contrast variation in SAXS emerges as a powerful characterisation technique for polymeric colloids, which can determine their size and density in a traceable way. For instance, the accuracy in the density information achieved with the density gradient technique is remarkable and extends along a rather large density range of polymers.

Since contrast variation in SAXS is very sensitive to small electron density differences in the colloid morphology, the applicability of this method to investigate the inner structure of 3 different particles has been discussed. This is of paramount importance in polymeric particle characterisation because the direct observation by imaging techniques is inadequate for this purpose. In fact, the detection of core-shell structures in polymeric colloids appears as essential for understanding the possible processes occurring during the particle formation, e.g. the consequences of emulsion polymerization synthesis. 

Furthermore, different evaluation approaches to contrast variation SAXS data are examined in detail. The isoscattering point framework is found to be of easy utilization and very appropriate for spherical and quite monodisperse colloids. On the other hand, the calculation of the scattering shape factor arises as a precise sizing technique which can additionally provide an insight into the particle shape, although a high number of measurements with different contrasts and an accurate calibration of the system are required.

These results were compared successfully with other techniques. In particular, SAXS measurements of the density of these colloids are in excellent agreement with those performed by DCS. The use of a novel DCS setup is also shown, which makes use of a centrifuge disc where the colloids float through a gradient of higher density, in contrast to a standard setup where the particles typically sediment. The use of the two complementary DCS configurations allowed the simultaneous determination of both the size and density of polymeric colloids consistently with the SAXS results. Both ensemble techniques presented in this work arise as powerful methods which can describe simultaneously the density and size distribution of polymeric particles at the nanoscale.

\section{chap6}

The results presented in this section demonstrate that it is possible to determine the size of complicated nanoparticles relevant in nanomedicine with continuous contrast variation in SAXS. This technique has been used to characterize a great variety of systems in the nanoscale such as the PEGylated liposomal nanodrug Caelyx\textregistered, empty liposomal nanocarriers, human lipoproteins \textcolor{red}{and protein-coated polymeric nanoparticles.}

In the case of Caelyx\textregistered, by means of an iso-osmolal density gradient, the position of the isoscattering point was measured whereby the size of the liposomal drug was determined with this model-free approach. Supplemented by the model fitting of the so called \emph{shape factor} of the liposomes, the size was also obtained from an independent evaluation procedure and an average size of (69 $\pm$ 5) nm was obtained. This size is smaller than the value measured by DLS, which can be attributed to the fact that the contrast variation SAXS determines the size of the liposomes impermeable to the contrast agent, i.e. the outer PEG layer of the liposomes is not probed. This demonstrates that the combination of SAXS with DLS can reveal the difference between the hydrodynamic diameter and the "core" size of the nanocarrier, which is related to the thickness of the PEG-layer in case of the stealth liposomes. Moreover it is shown that by means of the shape factor fitting, complementary information about the shape of the nanocarrier can be obtained. Additionally, it was found that the average electron density of the liposomal doxorubicin was higher than that of the empty PEGylated liposomes.

Besides, using an aqueous sucrose density gradient, it was possible to study the behavior of the liposomal drug carrier under different osmotic conditions. It was shown that an increasing osmolality of the buffer produces an osmotic shrinkage of the liposomal structure, although this structural deformation is reversible and does not affect the crystalline structure of the intraliposomal doxorubicin.

For comparison purposes with the liposomal doxorubicin system, the osmotic activity of empty liposomes was also investigated using aqueous sucrose. The distinguishable osmotic effects observed in PEGylated and plain liposomes arise from the different formation of the liposomes, which is influenced by the presence of PEG moieties in the preparation. The creation of multilamellar domains in the phospholipid layer was evaluated and the role of the PEG moieties in the membrane resilience was also investigated. The MLV structure of the plain liposomes show higher resiliance against osmotic pressure that the unilamellar membrane of the PEGylated vesicles. In the latter, the ULV structure shrinks due to the osmotic pressure and deforms the liposomes into obloid ellipsoids, creating a bilamellar structure at the outest part of the vesicles.

The continuous contrast variation technique was also used to determine the most distinctive traits of human lipoproteins: size and density, the latter being fundamental to classify them. The parameters obtained by means of a model-free analysis are in good agreement with the reported values in literature.

\textcolor{red}{Finally, the application of the technique on nanoparticles incubated in different concentrations of IgG reveals the difference between the IgG-shell and the hard protein-corona impenetrable for the solvent, probed with contrast variation SAXS. In addition, the use of complementary techniques such as DLS, SAXS and DCS show an increase of the protein-shell thickness with increasing concentration of the proteins during incubation. The SAXS contribution to this study requires of a model refinement for full consistency with the results measured for the bare PS-COOH particle.}
