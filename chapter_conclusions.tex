\chapter{Summary} 
\label{chap:conclusions}
This thesis demonstrates how continuous contrast variation in small-angle X-ray scattering (SAXS) by means of a density gradient capillary emerges as a powerful characterization technique for low-density nanoparticles. The technique has proven efficient on a great variety of systems relevant to nanomedicine such as polymeric nanocarriers, the PEGylated liposomal nano-drug Caelyx, empty liposomal nanocarriers and human lipoproteins. The possibility to collect an extensive data set of scattering curves in a short timespan and the ability to tune the contrast range during the experiment arise as clear advantages of the method. The scattering data acquired with this newly introduced technique has been analysed with complementary approaches to reveal a consistent insight into the size distribution and the inner structure of the suspended nanoparticles, resulting in the determination of the size and density of the nanoparticles in a traceable way. 

The application of the continuous contrast variation technique in SAXS to characterize low-density polymeric nanoparticles has been thoroughly reviewed in chapter~\ref{chap:density_gradient_SAXS}. Up to three different evaluation approaches were employed to determine the size of the PS-COOH nanoparticles. By using a model-free analysis of the experimental data based on the isoscattering point theory, an average particle diameter of $\left( 100.6 \pm 5.6 \right)$ nm was obtained, which was in very good agreement with the value obtained from a core-shell model fit of $\left( 99.4 \pm 5.6  \right)$ nm. 

The scope of the continuous contrast variation method as a sizing technique was revealed in chapter~\ref{chap:simultaneous_size_density} by the consistency of the results of the PS-Plain particles obtained with different evaluation approaches and techniques, like atomic force microscopy (AFM), differential centrifugal sedimentation (DCS) and transmission scanning electron microscopy (TSEM). Furthermore, different evaluation approaches to contrast variation SAXS data are examined in detail. The model-free isoscattering point framework is found to be of easy use and very appropriate for the size determination of spherical and quite monodisperse colloids. On the other hand, the calculation of the shape scattering function arises as a precise sizing technique which can additionally provide an insight into the particle shape, although a high number of measurements with different contrasts and an accurate calibration of the system are required.

Due to the high sensitivity of SAXS to small electron density differences in the colloid morphology, information about the heterogeneous composition of the particles can be retrieved. For instance, the analysis of the Guinier region of the scattering curves performed in section~\ref{sec:guinier_analysis} showed that the radial inner structure of the PS-COOH particles consisted of a thin, more dense layer coating the polystyrene core. Complementing these results, the form factor fit presented in section~\ref{sec:coreshell_fit} revealed that the core component of the particle had exactly the same electron  density expected for polystyrene and the shell was composed of a compound with a density below that of PMMA. This observation is of paramount importance in polymeric particle characterization because the direct observation by imaging techniques is inadequate for this purpose. In fact, the detection of core-shell structures in polymeric colloids appears as essential for understanding the possible processes occurring during the formation of the particle, e.g. the consequences of emulsion polymerization synthesis or the segregation of components due to their different hydrophobicity.

\textcolor{red}{Besides, a high accuracy in the density information is achieved with the density gradient technique and extends along a rather large density range of polymers as shown in chapter~\ref{chap:simultaneous_size_density}}. For instance, SAXS measurements of the density of three different polymeric colloids are in excellent agreement with those performed by DCS, a technique extensively used in nanoparticle characterization. As reviewed in section~\ref{sec:KiskerResultsEvaluation}, the determination of the average electron density of the particle by different evaluation approaches proves the continuous contrast variation technique as a useful tool and an alternative to other techniques like analytical ultracentrifugation, isopycnic centrifugation or field-flow fractionation.

At this point, the performance of the continuous contrast variation in SAXS for the simultaneous size and density determination of low-density polymeric nanoparticles has been successfully proven. The technique has evident advantages in comparison to other contrast variation techniques in small-angle scattering like deuterated small-angle neutron scattering (SANS) or anomalous SAXS (ASAXS), but certain limitations do also arise, namely its restriction to low-density nanomaterials due to the relatively low electron densities achievable with standard contrast agents. Nevertheless, the importance of the technique has been justified with its application to multiple nanomaterials relevant to research fields like medicine or biology in chapter~\ref{chap:bio_applications}.

In the case of the nano-drug Caelyx, a liposomal formulation of doxorubicin coated with polyethylene glycol (PEG), the position of the isoscattering point was measured by means of an iso-osmolal density gradient whereby the size of the liposomal drug was determined with this model-free approach. Supplemented by the model fitting of the shape scattering function of the liposomes, the size was also obtained from an independent evaluation procedure and an average diameter of (67 $\pm$ 5) nm was determined. This size is smaller than the value measured by dynamic light scattering (DLS), which can be attributed to the fact that the contrast variation SAXS determines the size of the liposomes impermeable to the contrast agent, i.e. the outer PEG layer of the liposomes is not probed. This demonstrates that the combination of SAXS with DLS can reveal the difference between the hydrodynamic diameter and the "core" size of the nanocarrier, which is related to the thickness of the PEG-layer in case of the stealth liposomes. Moreover it is shown that by means of the shape scattering function fitting, complementary information about the shape of the nanocarrier can be obtained. Additionally, it was found that the average electron density of the liposomal doxorubicin was higher than that of the empty PEGylated liposomes.

Using an aqueous sucrose density gradient, it was possible to study the behaviour of the liposomal drug carrier under different osmotic conditions. It was shown that an increasing osmolality of the buffer produces an osmotic shrinkage of the liposomal structure, although this structural deformation is reversible and does not affect the crystalline structure of the intraliposomal doxorubicin. For comparison purposes with the liposomal doxorubicin system, the osmotic activity of empty liposomes was also investigated using aqueous sucrose. The distinguishable osmotic effects observed in PEGylated and plain liposomes arise from the different formation of the liposomes, which is influenced by the presence of PEG moieties in the preparation. The creation of multilamellar domains in the phospholipid layer was evaluated and the role of the PEG moieties in the membrane resilience was also investigated. The multilamellar structure of the plain liposomes shows higher resilience against osmotic pressure than the unilamellar membrane of the PEGylated vesicles. In the latter, the unilamellar vesicle shrinks due to the osmotic pressure and deforms the liposomes into obloid ellipsoids, creating a bilamellar structure at the outest part of the vesicles.

The continuous contrast variation technique was also used to determine the most distinctive traits of human lipoproteins: size and density, while the application of the technique on nanoparticles incubated in different concentrations of Immunoglobulin G (IgG) revealed that the large uncertainty associated to the diffuseness of the isoscattering points makes the contrast variation approach inappropriate for the accurate and traceable determination of the protein-shell thickness. Nevertheless, the use of complementary techniques such as SAXS, DLS and DCS shows an increase of the protein-corona thickness with increasing concentration of the proteins during incubation as expected.

%the difference between the IgG-shell and the hard protein-corona impenetrable for the solvent. 

The work presented in this thesis proves that the recently developed continuous contrast variation technique in SAXS extends the possibilities of the classic solvent contrast variation approach to unexpected new heights. The use of a density gradient capillary results in a virtually continuous range of available solvent electron densities and opens up new perspectives in the characterization of low-density nanoparticles in suspension.
